\chapter{EXPERIMENTAL OPTIMIZATION OF PLASMA-LIQUID INTERACTIONS: VHF SOURCE}
\label{chap:expt_opt}

\Cref{chap:basic_science} outlines fundamental modeling efforts aimed at describing the physical and chemical phenomena that occur in plasma-liquid systems. \Cref{chap:zapdos} outlines the tool we created in order to better conduct our modeling efforts. To date modeling has been used to gain a better qualitative understanding of transport processes in convective plasma-liquid systems (\cref{sec:plasfree_model}) and to explore the effects of key interfacial parameters on plasma properties (\cref{sec:plasliq}). Model geometries have been based on relatively simple experimental set-ups (point-to-plane corona discharge for \cref{sec:plasfree_model}) or simplified to one dimension as in \cref{sec:plasliq}. However, the groundwork has been laid to model more complex plasma-liquid geometries and exotic electromagnetic fields. Such models will be used to describe the physiochemical properties observed in the complex experimental configurations described in this chapter. This chapter outlines plasma-liquid geometries that exhibit increasing degrees of plasma-liquid contact. In \cref{sec:base} we describe our base experimental configuration: a very high frequency (VHF) atmospheric plasma source that is pointed down into a reservoir of water such that the end of the plasma column is in direct contact with the water surface. In \cref{sec:spray} we discuss spraying water droplets directly through the plasma. After discussing the typical electrodes used on the VHF source in \cref{sec:electrodes}, we introduce in \cref{sec:water_electrodes} a completely novel design where the VHF source is pointed upward and water is pumped through the center of the inner conductor to form a water layer on top of the powered electrode. Finally, in \cref{sec:aq_chem} we show experimental measurements of aqueous chemistry generated by our plasma-liquid systems. We hope to reproduce our experimental measurements in \cref{sec:aq_chem} using a future combination and extension of the models and code described in \cref{chap:basic_science,chap:zapdos}.

\section{Description of NCSU VHF Source}
\label{sec:VHF}

A detailed description of the NCSU VHF source is given in \cite{byrns2012vhf}; a summary of the design is given below. The source is powered by a 3.5 kW 162 MHz generator (Advanced Energy Ovation 35162). The generator has a termination impedance of 50 $\Omega$ and is connected to the plasma source using a 50 $\Omega$ high-power coaxial cable. A directional coupler located at the output of the generator is used to track forward and reflected power. Source impedance matching is achieved using tuned stub matching. At the connection of the RF power cable and the plasma source, the RF signal is split towards load and ground terminations (see \cref{fig:batch_scheme}), the input and terminations are joined by a coaxial transmission line formed by aluminum inner and outer conductors. The inner diameter of the source coax is 2.25 cm and the outer diameter is 5.25 cm. With air as the feed gas, the coaxial structure's characteristic impedance is 51.7 $\Omega$. The length of the load and ground terminations are both variable, giving two degrees of freedom for matching. The last 3.8 cm of the inner conductor at the load termination is flared to a 3.5 cm diameter to assist in plasma ignition; the flared conductor piece is often hereafter referred to as the powered electrode. After ignition, a plasma column is observed in front of the powered electrode. It is speculated that the high driving frequency of the discharge creates a ballasting effect that prevents thermal arcing of the discharge. Ballasting occurs because increasing electron density actually increases the bulk plasma resistance, creating a negative feedback loop that tends to stabilize the glow discharge. \cite{byrns2012vhf}

\section{Base Set-up for Water Treatment}
\label{sec:base}

The experimental set-up shown in \cref{fig:batch_scheme} is known as the ``batch'' set-up. It was the first plasma-water configuration explored by the group. Originally it was intended for degradation of perfluorinated compounds like perfluorooctanesulfonic acid (PFOS) and perfluorootanoic acid (PFOA). It turned out that the batch configuration was unable to degrade these persistent chemicals; however, in the process it was discovered that the configuration generated large amounts of NO$_x$, mostly NO$_3^-$, in the aqueous phase. Generation of nitrogen and oxygen species (RONS) in solution by plasmas is now a well-known phenomenon in the plasma-liquid community; however, at the time it was a novel discovery for our group. Recognizing that aqueous nitrogen, specifically NO$_3^-$, is a key component in fertilizer, we were motivated to begin a study in collaboration with the horticulture department of plant fertilization using plasma activated water (PAW). This study is outlined in \cref{sec:fertigation}. Later, the batch configuration was used in exploration of dioxane degradation; this is discussed in \cref{sec:dioxane}.

Depending on the application, delivered power to the plasma for the batch configuration ranges between 350 and 1000 W. Many different gases are used, including air, argon, helium, nitrogen, and carbon dioxide. Gas flow rates range from 2-5 standard cubic feet per minute. The gap distance between the powered electrode and the water surface range from a few milimeters to a couple centimeters, with larger gap distances typically used in combination with larger gas flow rates in order to avoid splashing of the electrode and extinguishing of the plasma. Water treatment volumes typically span 100-500 mL for persistent chemical studies up to several liters for PAW generation in the fertigation experiments.

\begin{figure}[htbp]
  \centering
  \includegraphics[width=0.9\textwidth]{Figure1_word.pdf}
  \caption{Schematic of the atmospheric plasma source and batch water treatment set-up}
  \label{fig:batch_scheme}
\end{figure}

\section{Spray-through Design}
\label{sec:spray}

One way thought to increase plasma-water interaction is to directly introduce water droplets into the active plasma region. We hypothesize that there are two good reasons for doing this. Firstly, it is reasoned that water droplets passing through the core of the plasma as opposed to the edge or afterglow are exposed to greater densities of electrons, ions, and reactive radical species. Secondly, by breaking the water volume into droplets, the surface-to-volume ratio is increased, increasing the rate of mass transfer of plasma species into the aqueous phase. Two different configurations are used to explore these concepts; they are outlined in the following subsections.

\subsection{Spray Bottle}
\label{sec:spray_bottle}

The easiest way to achieve a droplet configuration is to take the batch set-up (see \cref{fig:batch_scheme}) and remove the beaker of water under the coaxial plasma source. Then after turning the plasma on, a greenhouse sprayer is used to pass droplets radially through the plasma; a beaker is used to catch the droplets after passage through the plasma. A summary of the configuration is shown in \cref{fig:spray_scheme}.

\begin{figure}[htbp]
  \centering
  \includegraphics[width=0.9\textwidth]{Figure6_word.jpg}
  \caption{Set-up for introducing water directly into the active plasma region.  A greenhouse sprayer injects water from the side of the plasma source; water is collected in a beaker on the other side}
  \label{fig:spray_scheme}
\end{figure}

A comparison of batch and greenhouse sprayer configurations for generation of nitrate in solution per unit energy is shown in \cref{fig:nitro_compare_power_design}. It is found that in general the greenhouse sprayer configuration performs more favorably than the batch treatment design. This is especially clear at higher powers. Moreover, the performance of the greenhouse sprayer configuration appears to improve with increasing power delivered to the plasma. However, increasing plasma power also has some negative effects. One is an increased rate of erosion of the powered electrode. A second negative consequence is increased reflected power back to the generator during plasma instabilitiesarising from the water droplets. Both of these effects decrease the lifetime of the design; decreasing the lifetime of the generator is particularly undesirable because of its cost.

\begin{figure}[htbp]
  \centering
  \includegraphics{Figure22_word.png}
  \caption{Comparison between batch and spray treatment methods using mmol of nitrate generated per kJ of electrical energy as the figure of merit.  For lower powers batch treatment is more energetically efficient for nitrate generation.  For higher powers, spray treatment is more efficient.}
  \label{fig:nitro_compare_power_design}
\end{figure}

What ultimately curtails further investigation of the spray-through design is the instability of the plasma. The sprayer must be placed such that droplets do not touch the surface of the powered electrode or else the plasma is immediately extinguished. Moreoever, even if the sprayer is properly placed and the electrode is not wetted, the plasma actively trys to avoid the droplet stream. This may occur for several reasons. Firstly, a much higher electric field must be applied to a dense liquid as opposed to a gas to create or sustain a discharge. Secondly, highly oxidative species like OH and OH$_2$ originating from the liquid phase can scavenge electrons. Typically even in the most optimized sprayer set-up, the plasma extinguishes after a few tens of seconds. Compare this with the batch set-up in which water can be treated continuously for multiple hours.

\subsection{Built-in Nozzle}

\begin{figure}[htbp]
  \centering
  \includegraphics{nozzle_diagram.png}
  \caption{Schematic of the nozzle electrode spray-through configuration}
  \label{fig:nozzle}
\end{figure}

\Cref{fig:nozzle} shows a schematic of the nozzle electrode experimental design. In terms of plasma-liquid contact, the concept is very similar to \cref{fig:spray_scheme} except the droplets are introduced vertically through the VHF source's inner conductor. Unfortunately, the nozzle electrode design suffers from the same pitfall as the greenhouse sprayer design. During operation, the plasma actively avoids the water droplets, moving with the cyclonic flow of the gas feed around the outside of the droplet stream. It is speculated that the droplet stream may form a partial Faraday cage inhibiting the discharge. Additionally, electronegative species like OH and OH$_2$ originating from the liquid phase may scavenge electrons in a manner analogous to the spray-through configuration of \cref{sec:spray_bottle}. On top of plasma instability originating from liquid interactions, the surface non-uniformity introduced by the nozzle on the electrode leads to faster surface erosion.

\section{Base electrode designs}
\label{sec:electrodes}

As mentioned in \cref{sec:spray_bottle}, plasma erosion of the source's powered electrode can occur, especially at higher powers. Evidence of this erosion can be seen both with the naked eye and in the optical emission spectrum of the discharge. \Cref{fig:OES_alum_damage} shows the presence of an atomic aluminum line at 395 nm and several AlO bands between 425 and 575 nm. Visually, this emission manifests itself as an intense bright blue. \Cref{fig:alum_damage_full} shows plasma color during normal operation, plasma color during aluminum damage, and the resulting appearance of the electrode after significant erosion.

\begin{figure}[htbp]
  \centering
  \includegraphics{damaged_aluminum_OES_spectrum.png}
  \caption{OES spectrum of plasma damaged aluminum electrode}
  \label{fig:OES_alum_damage}
\end{figure}

\begin{figure}[htbp]
  \centering
  \begin{subfigure}{.3\textwidth}
    \centering
    \includegraphics[width=\textwidth]{happy_plasma.jpg}
    \caption{Normal plasma color}
  \end{subfigure}
  \begin{subfigure}{.3\textwidth}
    \centering
    \includegraphics[width=\textwidth]{full_size_alum_electrode_damage.jpg}
    \caption{Image of aluminum electrode after plasma erosion}
  \end{subfigure}
  \begin{subfigure}{.3\textwidth}
    \centering
    \includegraphics[width=\textwidth]{argon_plasma_treating_dioxane.png}
    \caption{Plasma color during aluminum pitting}
  \end{subfigure}
  \caption{Normal vs. abnormal plasma glows}
  \label{fig:alum_damage_full}
\end{figure}

The plasma damage to the electrode can be investigated more closely using Secondary Electron Microscopy (SEM) and Energy Dispersive X-ray Spectroscopy (EDS). Even with a 1mm zoom (\cref{fig:alum_damage_1mm}), the growth of a damage layer is evident. Taking an EDS measurement of the clean aluminum yields the spectrum shown in \cref{fig:EDS_clean_alum}. Unsurprisingly, the spectrum shows almost pure aluminum with a trace of magnesium. An EDS scan of the damaged aluminum portion, however, reveals the growth of substantial carbon and oxygen peaks (\cref{fig:EDS_damaged_alum}). The oxidation is unsurprising considering the flow gas is often compressed air and the ambient environment is also air (also consistent with the OES spectrum (\cref{fig:OES_alum_damage})). The carbon could be coming from oils/hydrocarbons present in the compressed air feed.

\begin{figure}[htbp]
  \centering
  \includegraphics{1mm_zoom_tilt_alum_electrode_damage.png}
  \caption{SEM image of aluminum electrode after plasma erosion. 1mm zoom. 45 degree tilt.}
  \label{fig:alum_damage_1mm}
\end{figure}

\begin{figure}[htbp]
  \centering
  \includegraphics[width=0.9\textwidth, height=0.9\textheight, keepaspectratio]{clean_alum_eds.png}
  \caption{Energy dispersive X-ray spec (EDS) for clean aluminum electrode}
  \label{fig:EDS_clean_alum}
\end{figure}

\begin{figure}[htbp]
  \centering
  \includegraphics[width=0.9\textwidth, height=0.9\textheight, keepaspectratio]{damaged_alum_eds.png}
  \caption{Energy dispersive X-ray spec (EDS) for plasma eroded aluminum electrode}
  \label{fig:EDS_damaged_alum}
\end{figure}

In an attempt to prolong the lifetime of the powered electrode, metals other than aluminum are considered. A relatively inexpensive choice is brass. Overall, brass performs much better than aluminum. Between 300-700 W, there is no plasma-metal interactions observed with OES or presence of pitting when the plasma is turned off. Typically aluminum begins to erode around 560 W. When the brass electrode is run between 700-1000 W, plasma-metal interactions are evinced by a plasma color change as well as an increase in the intensity of the emitted light. A comparison of the plasma OES with and without metal interactions is shown in \cref{fig:OES_brass_damage}. The 560 W spectrum shows a more or less normal air plasma spectrum: NO bands between 230 and 290 nm (along with their 2x peaks around 500nm) and an OH band around 310 nm. However, the 945 W spectrum is dominated by sharp copper and zinc atomic lines. Despite the presence of copper and zinc in the discharge emission, no visual damage appears on the electrode surface when operated between 700 and 1000 W. However, if the power is raised too much over 1000 W, surface pitting and scarring analagous to the damage on the aluminum electrode are observed (see \cref{fig:brass_damage_full}).

\begin{figure}[htbp]
  \centering
  \includegraphics[width=0.9\textwidth]{damaged_brass.jpg}
  \caption{Image of brass electrode after plasma erosion}
  \label{fig:brass_damage_full}
\end{figure}

\begin{figure}[htbp]
  \centering
  \includegraphics{damaged_brass_OES_spectrum.png}
  \caption{Top OES spectrum shows plasma emissions during normal operation with the brass electrode. The bottom spectrum shows emissions that occur during brass damage}
  \label{fig:OES_brass_damage}
\end{figure}

\section{Water Electrodes}
\label{sec:water_electrodes}

An ideal plasma-liquid geometry has to provide both maximum interfacial contact between reactive plasma species and the liquid phase as well as system components that are resistent to plasma corrosion. Unfortunately, none of our previous configurations realize this goal. However, by utilizing the unique nature of the VHF source and recognizing that the entire coaxial structure is DC grounded, we can do something rather novel. We can apply a liquid layer to the surface of the powered electrode without worry of causing a short circuit. With this configuration, shown in \cref{fig:water_electrode_scheme}, the treated water is exposed to the most reactive part of the plasma. Both ion and electron fluxes to the water surface are anticipated to be much higher than in the batch configuration. Additionally, powered plasma-facing solid surfaces are completely eliminated from the geometry. The liquid surface is forever renewable and does not erode. This reduces system cost as well as experimental down-time.

\begin{figure}[htbp]
  \centering
  \includegraphics[width=0.9\textwidth]{water_electrode_geometry.png}
  \caption{Representative experimental set-up for using a ``water'' electrode}
  \label{fig:water_electrode_scheme}
\end{figure}

The actual design of the water electrode can be seen in \cref{fig:water_electrodes_image}. The electrode of most utility, the ``pure'' water electrode, is shown on the right. The pure water electrode has no powered metal surfaces facing the plasma; the powered plasma-facing surface is 100\% water. If some plasma-metal contact is desired, for instance if the contact favorably modifies some plasma or liquid application variable, then the ``annular'' electrode shown on the left can be used.

\begin{figure}[htbp]
  \centering
  \includegraphics[width=0.9\textwidth]{water_and_annular_electrodes.jpg}
  \caption{Image of the two versions of ``water'' electrodes. The ``annular'' version still allows a small metallic area of plasma contact. In the ``pure'' version, the plasma has no metallic content with the powered electrode. The powered surface is entirely composed of water.}
  \label{fig:water_electrodes_image}
\end{figure}

\subsection{Circuit Analysis}
\label{sec:circuit}

In order to better understand the coupling of the RF power to the plasma-liquid system, it is useful to construct a circuit model. The first question the circuit model should answer is whether conduction current coming from the inner conductor propagates along the water electrode surface or the underlying aluminum. This is done by comparing the relative resistances presented by the water and aluminum surfaces, treating both as conductors. The resistance is calculated using the relationship:

\begin{equation}
  R = \frac{L\rho}{A}
  \label{eq:skin_R}
\end{equation}

where $R$ is the resistance, $L$ is the length of the conducting surface, $\rho$ is the resistivity of the medium, and $A$ is the cross-sectional area through which the conduction current can flow. For current propagation across the top face of the cylindrical electrode, we approximate $L$ by the radius $r$ of the electrode, and $A$ by the product of the skin depth $\delta$ and the radius $r$. The skin depth $\delta$ is calculated using:

\begin{equation}
  \delta = \sqrt{\frac{2\rho}{\omega\mu_B}}
  \label{eq:skin_depth}
\end{equation}

where $\omega$ is the driving frequency in radians/s and $\mu_B$ is the material permeability, set equal to $\mu_0=4\pi\times10^{-7}$. For aluminum, $\rho$ is set equal to its literature value at 20$^{circ}$ Celsius, $2.82\times10^{-8}\Omega m$. A typical tap water conductivity of 50 mS/m is used to calculate the resistivity of water, $\rho = \frac{1}{\sigma}$. The corresponding skin depth for water at 162 MHz is .18 m, which is significantly larger than the milimeter depth of the water layer. As a consequence, $A$ for \cref{eq:skin_R} is calculated with $\delta$ = 1 mm. With these definitions,tThe resistance of the water surface to conduction current is calculated to be 20 thousand $\Omega$ at 162 MHz. The skin depth of aluminum at 162 MHz is 6 $\mu$m. The corresponding resistance to conduction current is 4 m$\Omega$ at 162 MHz. One can ask whether plasma modification of the water surface might substantially decrease the water resistance; however, because the mobility of electrons in water is so much lower than their gaseous mobility, the effect of the plasma is nowhere near enough to overcome the seven order of magnitude difference in resistance between water and aluminum. This analysis suggests that all of the conduction current coming from the feed line propagates along the underlying aluminum electrode as opposed to the water surface. A frequency analysis shown in \cref{fig:surf_propag} demonstrates that conduction current will likely flow through the underlying aluminum regardless of the device operating frequency.

\begin{figure}[htbp]
  \centering
  \includegraphics[width=0.9\textwidth]{alum_vs_water_propagation.eps}
  \caption{Resistance to flow of conduction current for aluminum and water for a range of frequencies. Vertical, black, dotted line indicates the 162 MHz operating frequency of the NCSU source. Aluminum is orders of magnitude less resistive for all frequencies considered; consequently the current propagating along the feed line is likely to prefer the underlying aluminum electrode over the water surface.}
   \label{fig:surf_propag}
\end{figure}

The demonstration that conduction current likely does not propagate along the water \textit{surface} means that current, most likely in the form of displacement current, must pass through the water \textit{volume}. The question then becomes: what is the relative split in dissipated power between the water and plasma? Where is the potential drop occuring? These are answered by treating the water volume and plasma as lossy dielectrics. We define the medium dielectric constant by:

\begin{equation}
  \epsilon = \epsilon_r\epsilon_0\left(1 - \frac{\omega_c^2}{\omega\left(\omega - \nu j\right)}\right)
  \label{eq:epsilon}
\end{equation}

where $\epsilon_r$ is the relative dielectric constant, $\epsilon_0$ is the permittivity of free space, $\omega_c$ is the characteristic frequency coming from oscillatios of free charges in the medium, $\omega$ is again the driving frequency, $\nu$ is the rate of collisions of charges with background molecules, and j is the imaginary number $\sqrt{-1}$. For the plasma, $\epsilon_r=1$; for the water, $\epsilon_r=80$. Because the free electrons are much lighter than their corresponding ions, $\omega_c$ in the plasma is essentially equal to the plasma electron frequency $\omega_{pe}$. In the water, $\omega_c$ is calculated assuming sodium and chloride charge carriers. $\omega_c$ in both plasma and water is calculated using the equation:

\begin{equation}
  \omega_c = \sqrt{\frac{e^2n_0}{\epsilon_0 m}}
  \label{eq:omega_c}
\end{equation}

where $e$ is the Coulombic charge, $n_0$ is the number density, and $m$ is the particle mass. For the plasma, $\nu$ is calculated using:

\begin{equation}
  \nu = n_g\sqrt{\frac{\pi \alpha e^2}{m_e\epsilon_0}}
  \label{eq:nu_plasma}
\end{equation}

where $n_g$ is the background gas density and $\alpha$ is the polarizability equal to $2.1\times10^{-29}m^3$ for air. For the water, $\nu$ is determined via

\begin{equation}
  \nu = \frac{e}{\mu m}
  \label{eq:nu_water}
\end{equation}

where $\mu$ is the mobility calculated with Einstein's relation

\begin{equation}
  \mu = \frac{D e}{k_b T}
  \label{eq:einstein}
\end{equation}

where D is the diffusivity of sodium and chloride ions, equal to $2\times10^{-9}m^2s^{-1}$ \cite{morrow2006time}, $k_b$ is Boltzmann's constant, and T is the temperature of the water (assumed equal to 300 K). Once the medium dielectric constant $\epsilon$ is calculated, the medium admittance is computed using the approximation of a parallel plate capacitor:

\begin{equation}
  Y = \frac{\omega \epsilon A j}{d}
  \label{eq:admittance}
\end{equation}

Finally, the impedance $Z$ is computed using the simple relation, $Z = \frac{1}{Y}$. For a lossy dielectric, the impedance $Z$ is complex, e.g. $Z = R + Xj$ with R the resistance and X the reactance. \Cref{fig:resist_compare} compares the plasma and water resistance over a wide range of frequencies. Over the entire range of frequencies presented the water resistance is < 1\% of the plasma resistance. At the operating frequency of the VHF source, the water resistance is close to six orders of magnitude less than the plasma resistance, suggesting that virtually all of the RF power is dissipated in the plasma.

\begin{figure}[htbp]
  \centering
  \includegraphics[width=0.9\textwidth]{Plasma_vs_water_resistance.eps}
  \caption{Comparison of plasma and water resistances over a range of operating frequencies. Over the whole domain, the water resistance is < 1\% the plasma resistance. At 162 MHz they differ by six orders of magnitude, suggesting that all of the RF power is dissipated in the plasma as opposed to the water.}
  \label{fig:resist_compare}
\end{figure}

\Cref{fig:react_compare} compares the magnitude of the plasma and water reactance as a function of frequency. For both mediums, the reactance is negative for all frequencies, consistent with capacitive behavior. (Lower gas pressures lead to more inductive behavior.) For frequencies < 1 MHz, the plasma and water reactance are roughly equivalent. Beyond 1 MHz, the magnitude of the water reactance decreases while the magnitude of the plasma reactance continues to increase until roughly 80 MHz beyond which it begins to decline. \Cref{fig:imped_mag} compares the magnitude of the impedance ($|Z| = \sqrt{R^2 + X^2}$) for plasma and water for a range of frequencies. The impedance magnitude for water is determined primarily by its resistive component below 1 MHz and by its reactive component above 1 MHz. The behavior for the plasma is similar except the transition occurs around 30 MHz. The result is that the plasma impedance magnitude is always a couple of magnitudes larger than the water impedance magnitude for all frequencies, ensuring that the majority of the potential drop always occurs across the plasma as opposed to the water.

\begin{figure}[htbp]
  \centering
  \includegraphics[width=0.9\textwidth]{Plasma_vs_water_reactance.eps}
  \caption{Comparison of plasma and water reactance magnitude for a range of frequencies. Magnitudes are roughly equivalent up to 1 MHz, where water reactance magnitude begins to decline. Plasma knee occurs around 80 MHz.}
  \label{fig:react_compare}
\end{figure}

\begin{figure}[htbp]
  \centering
  \includegraphics[width=0.9\textwidth]{Plasma_vs_water_impedance_mag.eps}
  \caption{Impedance magnitudes for plasma and water as a function of frequency. Plasma impedance magnitude is significantly larger over the whole frequency domain.}
  \label{fig:imped_mag}
\end{figure}

\subsection{Optical Emission}
\label{sec:OES}

\Cref{fig:annular_vs_water_oes} compares typical OES spectra obtained for annular and pure water electrodes running at 700 W. Evidence of plasma-metal contact with the annular electrode is evident in the presence of aluminum atomic lines and AlO molecular bands. Additionally, there is a sodium line from sputtering of the tap water. The pure water electrode spectrum is much less intense and consists only of OH bands.

\begin{figure}[htbp]
  \centering
  \includegraphics{annular_vs_water_electrode_oes.png}
  \caption{Comparison of OES spectra for annular and pure water electrodes}
  \label{fig:annular_vs_water_oes}
\end{figure}

\Cref{fig:pow_sweep_water} shows the effect of increasing power on the optical emission spectrum with the pure water electrode. Because the plasma is in immediate contact with the water surface and not any solid surfaces, the device can be operated at much higher powers. Whereas with a metal electrode the source cannot be run at powers much greater than 700 W without significant damage to the electrode, the pure water electrode can be run up to 1155 W. The only reason that the device cannot be operated at even higher powers is because of the increased difficulty in matching impedances using the main and stub lines; reflected power becomes high enough to potentially damage the generator.

Up above 1000 W, aluminum atomic lines and AlO bands become apparent (\cref{fig:pow_sweep_water}). The presence of the aluminum associated emissions is interesting because the metal is removed from the gaseous discharge region by the few milimeter thick water layer; this probably explains why the intensities are significantly below that of discharges with direct plasma-metal contact (see \cref{fig:annular_vs_water_oes}). However, the existence of any aluminum lines at all implies that the discharge is penetrating the aqueous phase to reach the metal. This is perhaps indirect evidence of high charged particle fluxes to the water electrode surface, suggesting that the pure water electrode design is a good one for maximizing interactions between the plasma and aqueous phases.

\begin{figure}[htbp]
  \centering
  \includegraphics{water_electrode_power_sweep_oes.png}
  \caption{OES spectra showing power sweep with pure water electrode. Relatively small aluminum peaks grow in at very high powers. Cause of aluminum peak deformations unknown, but speculation is that it could be signal attenuation by the water layer}
  \label{fig:pow_sweep_water}
\end{figure}

% Note that there is a lot of OH rotational measurement as well as nitrite and nitrate concentration measurements for the water electrodes that are in my ICOPS 2014 presentation. I think the data looks like a bunch of garbage with no clear trends so I'm omitting it for now. However, if I need more material, I can perhaps come back to it.

\subsection{Absorption Work}
\label{sec:absorption}

Some idea of the magnitude of OH produced by the water electrode geometry can be gained by performing absorption spectroscopy. A picture of the experimental set-up is shown in \cref{fig:expt_abs}. The diameter of the plasma column is approximately 2 cm. We pass a broadbeam light source through slits cut in the outer conductor of the VHF shource. Mirrors on each slit side can be used to route the light beam through the plasma column for a total of up to 4 passes and a path-length up to 8 cm. After passing through the plasma, the beam enters an optical fiber connected to a high resolution spectrometer. The spectrum of the broadband light source in the absence of plasma is presented as series ``no-plasma'' in \cref{fig:raw_abs}. When the plasma is turned on, there is significant attenuation of the broadband signal in the region of the OH X-A electronic transition with just a single pass through the plasma as shown in \cref{fig:raw_abs}. The y-axis data for \cref{fig:net_abs} is calculated using:

\begin{equation}
  1 - \frac{I-I_p}{I_0}
  \label{eq:absorp}
\end{equation}

where $I$ is the spectrum taken with both the broadband light source and plasma, $I_p$ is the spectrum of plasma only, and $I_0$ is the spectrum of just the broadband light source. The fingerprint of the OH X-A transition is evident.

\begin{figure}[htbp]
  \centering
  \includegraphics[width=\textwidth]{abs_setup.png}
  \caption{Expermental set-up for absorption spectroscopy with the water electrode.}
  \label{fig:expt_abs}
\end{figure}

\begin{figure}[htbp]
  \centering
  \includegraphics[width=\textwidth]{raw_absorption_spec.png}
  \caption{Raw optical spectra in the OH wavelength region for different plasma powers and a path length of 2 cm (single pass). The series ``No plasma'' shows the light intensity of just the broadband light source. The other series clearly show the absorption of light by gaseous OH radicals. Highlighted region shows the area of integration for later calculation of OH densities}
  \label{fig:raw_abs}
\end{figure}

\begin{figure}[htbp]
  \centering
  \includegraphics[width=\textwidth]{net_absorption_spec.png}
  \caption{Net results obtained by subtracting plasma absorption spectra from broadband light source spectrum and normalizing. Obvious OH X-A transition fingerprint. Highlighted region shows the area of integration for later calculation of OH densities}
  \label{fig:net_abs}
\end{figure}

We can approximately calculate the density of ground-state OH in the plasma using Beer-Lambert's law:

\begin{equation}
  \frac{I-I_p}{I_0} = exp\left(-\sigma(\lambda) L N\right)
  \label{eq:beer-lambert}
\end{equation}

where $\sigma$ here is the cross section in units of area for absorption of light of wavelength $\lambda$, $L$ is the path length for absorption, and $N$ is the density of the absorbing species, OH in this case. Because the OH(X-A) transition is spread over a variety of vibrational and rotational states, one can choose a wavelength range to integrate over. As indicated by the highlight in \cref{fig:raw_abs,fig:net_abs}, the wavelengths spanning the P branch, roughly 309-309.5 nm, are chosen for integration. When performing the integration,

\begin{equation}
  \int_{309}^{309.5} \left(1 - \frac{I-I_p}{I_0}\right)d\lambda = \int_{309}^{309.5}\left(1 - exp\left(-\sigma(\lambda) L N\right)\right)d\lambda
  \label{eq:int_unsimplified}
\end{equation}

the integrand on the RHS can be simplified because the argument of the exponential is small, allowing the approximation $exp(-x)\approx 1 - x$. After performing this substitution, the concentration of OH can be calculated using:

\begin{equation}
  N = \frac{\int_{309}^{309.5}\left(1 - \frac{I-I_p}{I_0}\right)d\lambda}{L\int_{309}^{309.5}\sigma(\lambda)d\lambda}
  \label{eq:int_simp}
\end{equation}

\Cref{fig:OH_dist} shows the density of OH as a function of distance from the powered electrode for powers of 455, 560, and 665 W. OH densities are on the order of 10$^{15}$ cm$^{-3}$. The density decreases monotonically with increasing distance from the powered electrode.

\begin{figure}[htbp]
  \centering
  \includegraphics[width=\textwidth]{distance_vs_OH_density.png}
  \caption{OH density versus position from the powered electrode for a variety of powers}
  \label{fig:OH_dist}
\end{figure}

\Cref{fig:OH_pow} shows the OH density as a function of power at a distance of 8 cm from the powered electrode. From 455 to 700 W the OH density increases linearly from a minimum value of 2x10$^{14}$ cm$^{-3}$ to a maximum value of 1.2x10$^{15}$ cm$^{-3}$. These high OH densities could be valuable in applications requiring a high degree of oxidative stress, such as in various fields of plasma medicine and pollutant degradation. Concentrated OH could be a key player in the degradation of persistent chemicals like dioxane and PFOS, as described in \cref{sec:pollutant}.

\begin{figure}[htbp]
  \centering
  \includegraphics[width=\textwidth]{OH_dens_vs_power.png}
  \caption{OH density versus power 8 cm from the powered electrode. Clear increasing trend of OH density with power}
  \label{fig:OH_pow}
\end{figure}

\section{Exploring Aqueous Chemistry Generated by Plasma-Liquid Interactions}
\label{sec:aq_chem}

In addition to plasma characterization with OES and absorption spectroscopy, additional research has focused on optimizing and understanding generation of nitrates and nitrites in aqueous solution.  Several variables have been explored, including power supplied by the 162 MHz generator, flow rate of the feed gas, type of interface between the plasma and water phases, and the effect of aqueous impurities, particularly basic species. The majority of experiments were performed using the experimental set-up shown in \cref{fig:batch_scheme}.  However, the greenhouse sprayer scheme shown in \cref{fig:spray_scheme} was also employed.  The number of impurities and basic species in water were controlled in two manners.  The first was the choice between distilled and tap water, with the former containing negligible impurities and the latter containing impurities found in Raleigh's municipal water supply; these impurities are summarized in \cref{tab:tap_water}.  The most relevant item in \cref{tab:tap_water} is the alkalinity, which comes primarily from the carbonate system.  At a pH of 8.4, it is reasonable to assume that the tap water alkalinity is completely due to bicarbonate. \cite{benjamin2014water} Using this assumption, the concentration of bicarbonate in tap water is .50 mmol/L.  The concentration of bicarbonate can also be directly controlled by adding measured amounts of NaHCO$_3$.   NaHCO$_3$ can be added pre- or post-exposure depending on the experiment.  The motivation for adding basic species like NaHCO$_3$ to solution is that they are known to react with dissolved NO and NO$_2$ to form nitrite. \cite{greenwood1984chemistry} Thus basic species concentrations can be a control knob for adjusting the nitrogen chemistry in PAW.

\begin{table}[htpb]
  \begin{center}
    \begin{tabular}{|c |c |}
      \hline
      pH & 8.4 \\\hline
      Free CO$_2$ & .23 \\\hline
      Total alkalinity (mg/L as CaCO$_3$) & 24.8 \\\hline
      Total hardness (mg/L as CaCO$_3$) & 24.4 \\\hline
      Total dissolved solids (mg/L) & 150 \\\hline
      Specific conductivity ($\mu$S/cm) & 225 \\\hline
      Iron (mg/L) & .01 \\\hline
      Manganese (mg/L) & .02 \\\hline
      Fluoride (mg/L) & .78 \\\hline
      Chloride (mg/L) & 13.3 \\\hline
      Silica (mg/L) & 8.12 \\\hline
      Silt density index (SDI) & 5.00 \\
      \hline
    \end{tabular}
  \end{center}
  \caption{Impurities in Raleigh tap water}
  \label{tab:tap_water}
\end{table}

At a gas flow rate of .14 m$^3$/min, an exposure time of 3 minutes, and a treatment volume of 150 mL distilled water, nitrate concentrations were determined for powers ranging from 385 to 630 W and are shown in \cref{fig:nitrogen_vs_energy}.  For a better comparison with spray treatment results shown in \cref{fig:nitrogen_vs_energy_spray}, the horizontal axis is defined in terms of the energy deposited in the plasma per mass of water exposed to the plasma.  The results indicate a general downward trend in nitrate concentration with respect to power and total energy deposition.   To decouple the effects of power and total energy deposition, a second experiment was conducted in which treatment times were varied with power in order to keep the total energy delivered to the plasma constant.  Consequently, whereas the exposure time was 3 minutes for a 420 W plasma, exposure time was only 2 minutes for a 630 W plasma for a constant plasma energy deposition of 75.6 kJ; for comparison with \cref{fig:nitrogen_vs_energy}, the energy deposited in the plasma per mass of exposed water was 504 kJ/kg.  Gas flow was again .14 m$^3$/min and water volumes were 150 mL.  Results from the second experiment are shown in \cref{fig:nitrogen_vs_power}.  Though the plasma energy deposition is constant, nitrate concentrations in both tap and distilled water samples decrease with increasing power, consistent with the trend in \cref{fig:nitrogen_vs_energy}.  Though no nitrite appears in distilled water samples, nitrite concentrations increase in tap water with increasing power.  The total nitrogen anion levels in tap and distilled water samples are within experimental error for powers between 420 and 560 W.

\begin{figure}[htbp]
  \centering
  \includegraphics{Figure17_word.png}
  \caption{Nitrate concentration in distilled water versus energy deposited in the plasma per mass of exposed water. No detectable amount of nitrite generated}
  \label{fig:nitrogen_vs_energy}
\end{figure}

\begin{figure}[htbp]
  \centering
  \includegraphics{Figure18_word.png}
  \caption{Nitrate and nitrite concentrations in tap water versus power.  Treatment times scaled such that for each power setting, total energy deposited in system is constant at 504 kJ/kg H2O}
  \label{fig:nitrogen_vs_power}
\end{figure}

Another variable explored was gas flow rate.  For an exposure of 3 minutes, a treatment volume of 150 mL distilled water, and a plasma power input of 420 W, nitrate concentrations were measured for flow rates of .08, .11, and .14 m$^3$/min and are recorded in \cref{fig:nitrogen_vs_flow}.  A factor of 4.9 improvement in nitrate concentration is observed between .08 and .14 m$^3$/min flow settings.  Again no detectable amount of nitrite was observed.

\begin{figure}[htbp]
  \centering
  \includegraphics{Figure19_word.png}
  \caption{Nitrate concentration in distilled water versus air flow rate. No detectable amount of nitrite generated}
  \label{fig:nitrogen_vs_flow}
\end{figure}

One variable with remarkable effects on nitrogen species concentration is the presence of basic species before plasma exposure and also addition of basic species after plasma exposure.  As mentioned in the experimental section and as will be touched on further in the discussion section, basic species are known to react with dissolved NO and NO$_2$ (which are formed in the plasma) to form nitrite. \cref{tab:bicarb} summarizes a series of experiments in which the effect of adding approximately 6 mmol/L of sodium bicarbonate before or after plasma exposure was observed on tap and distilled water substrates (200 mL volume).  In both distilled and tap water samples, adding sodium bicarbonate before plasma exposure produced significantly more nitrite than when it was added post-exposure, which in turn produced significantly more nitrite than when no bicarbonate was added at all.  For all three treatment schemes, tap water ended with more nitrite than distilled.  Nitrate trends were not as clear.

\begin{table}[htpb]
  \begin{center}
    \begin{tabularx}{\textwidth}{|c |c |c |X |c |}
      \hline
      \textbf{Nitrite (mmol/L)} & \textbf{Nitrate (mmol/L)} & \textbf{pH} & \textbf{Description} & \textbf{Sample reference \#} \\\hline
      .041 & 1.09 & 3.18 & Tap, no NaHCO$_3$ addition & 1 \\\hline
      2.43 & .795 & 7.99 & Tap, 5.71 mmol/L NaHCO$_3$ added pre-exposure & 2 \\\hline
      .617 & .981 & 7.68 & Tap, 6.19 mmol/L NaHCO$_3$ add post-exposure & 3 \\\hline
      .004 & .795 & 2.88 & Distilled, no NaHCO$_3$ addition & 4 \\\hline
      .854 & .273 & 8.02 & Distilled, 5.71 mmol/L NaHCO$_3$ added pre-exposure & 5 \\\hline
      .235 & 1.37 & 7.55 & Distilled, 6.67 mmol/L NaHCO$_3$ added post-exposure & 6 \\\hline
    \end{tabularx}
  \end{center}
  \caption{Dependence of nitrogen ionic species on water type and amount of NaHCO$_3$ in solution.  The sample \#'s are used as references in the discussion section}
  \label{tab:bicarb}
\end{table}

Another variable that was manipulated was the time between plasma exposure and post-exposure addition of NaHCO$_3$. \Cref{fig:nitrogen_vs_time_delay} shows that while the total molar concentration of ionic nitrogen species is a constant, increasing the time between plasma exposure and NaHCO$_3$ addition increases nitrate concentration and decreases nitrite concentration.

\begin{figure}[htbp]
  \centering
  \includegraphics{Figure20_word.png}
  \caption{Effect of time delay between plasma exposure and bicarbonate addition on nitrite and nitrate species concentrations}
  \label{fig:nitrogen_vs_time_delay}
\end{figure}

A fundamental change in the set-up of the system can be realized by removing the stagnant water volume from underneath the electrode and instead spraying the water substrate directly through the active plasma region as described in the experimental section and as shown in \cref{fig:spray_scheme}.  Some difficulty is experienced in maintaining the plasma during water spray operation.  The plasma actively attempts to avoid the region through which the water passes; if the water spray blankets the entire area which the plasma normally occupies, the discharge may extinguish.  However, if the plasma is maintained, the increased biphasic interaction is demonstrated by frequent orange light emission from excited sodium in tap water.  For this alternative geometry the effects of power and gas flow rate on nitrate uptake are in opposition to the trends witnessed for the stationary water phase geometry.   For one pass of distilled water through the active plasma region \cref{fig:nitrogen_vs_energy_spray} shows increasing nitrate uptake with increasing power for a gas flow rate of .11 m$^3$/min (no nitrite formed).  Instead of power on the x-axis, energy per kg of exposed water is used in order to enable a comparison to the results shown in \cref{fig:nitrogen_vs_energy}.  The concentration of nitrate generated in the water is an order of magnitude less in \cref{fig:nitrogen_vs_energy_spray} than it is in \cref{fig:nitrogen_vs_energy}, but the energy usage per kg of exposed water is also an order of magnitude less.  A more obvious comparison between spray and batch treatments can be done by combining \cref{fig:nitrogen_vs_energy,fig:nitrogen_vs_energy_spray} and plotting the amount of nitrate generated per energy usage as a function of power, as is done in \cref{fig:nitro_compare_power}.  The most efficient nitrate generation occurs at 700 W using spray treatment, yielding 4.6 μmols of nitrate per kJ.  However, based on the observed trends, even more efficient nitrate generation may be realized by continuing to increase power with spray treatment or by decreasing power with batch treatment.  \Cref{fig:nitro_vs_flow_spray} shows that spray treatment efficiency may also be improved by decreasing gas flow rate.

\begin{figure}[htbp]
  \centering
  \includegraphics{Figure21_word.png}
  \caption{Dependence of nitrate uptake on plasma energy deposition for water sprayed through the active plasma region.  Compare with results in Figure 17 for batch treatment}
  \label{fig:nitrogen_vs_energy_spray}
\end{figure}

\begin{figure}[htbp]
  \centering
  \includegraphics{Figure22_word.png}
  \caption{Comparison between batch and spray treatment methods using mmol of nitrate generated per kJ of electrical energy as the figure of merit.  For lower powers batch treatment is more energetically efficient for nitrate generation.  For higher powers spray treatment is more efficient.  Further investigation of batch process at lower powers and spray process at higher powers required to determine optimal process for nitrate generation}
  \label{fig:nitro_compare_power}
\end{figure}

\begin{figure}[htbp]
  \centering
  \includegraphics{Figure23_word.png}
  \caption{Dependence of nitrate uptake on gas flow rate for water sprayed through the active plasma region. Power = 560 W.}
  \label{fig:nitro_vs_flow_spray}
\end{figure}

The species concentration trends observed in \cref{fig:nitrogen_vs_energy,fig:nitrogen_vs_power,fig:nitrogen_vs_energy_spray} are believed to result from the dependence of electron density and gas temperature on delivered power and from the dependence of interfacial mass transfer on system configuration.   Consider the trend shown in \cref{fig:nitrogen_vs_energy,fig:nitrogen_vs_power} for the concentration of nitrate as a function of power and energy deposition for the case where the water surface is held stationary directly under the plasma.  As power increases, the amount of nitrate produced decreases.  It is possible that the increase in electron density that occurs simultaneously with increasing power creates a more reductive environment which enables more formation of  nitrite as evidenced in \cref{fig:nitrogen_vs_power} (oxidation state = +3) or other more reduced NOx forms such as NO (+2), NO$_2$ (+4), and N$_2$O (+1)  relative to nitrate (+5).  This analysis, however, is confounded by the trend observed in \cref{fig:nitrogen_vs_energy_spray} in which nitrate uptake increases with increasing power when water is sprayed through the active plasma region.  A tentative explanation is that when the water surface is held stationary below the plasma the outgoing convective flow of the feed gas restricts diffusion of water vapor into the plasma region, a restriction that is not present when water is directly injected into the discharge.  If water vapor is present in the active plasma region, an increase in power should correspond to an increase in hydroxyl radical formation because of an increase in the rate of electron-impact dissociation.  This should increase the oxidizing nature of the plasma and subsequently increase nitrate production; this is observed in \cref{fig:nitrogen_vs_energy_spray} for direct water injection.  While this logic may also extend to the case where the plasma hovers over the stationary water surface, it can be expected to occur to a much more limited degree compared to the direct injection case because the convective wind of the feed gas whisks water vapor away from the active plasma region.  Consequently there is not a sufficiently large increase in the oxidizing character of the plasma to offset the increase in reductive character due to electron density; the nitrate concentration then decreases with power as observed in \cref{fig:nitrogen_vs_energy,fig:nitrogen_vs_power}.

The argument presented in the previous paragraph also supports the trend shown in \cref{fig:nitro_vs_flow_spray}, where nitrate concentration decreases with increasing gas flow rate for the case of direct water injection.  An increase in gas flow rate decreases the residence time of gas molecules in the glow region, decreasing the gas temperature.  Decreasing gas temperature decreases the vaporization rate of liquid droplets, leading to a decreased concentration of hydroxyl in the plasma and a decreased ability to oxidize gaseous nitrogen species to nitrate.  This theory, however, contradicts the trend seen in \cref{fig:nitrogen_vs_flow} for stationary water where nitrate increases with flow rate.  One explanation is that the decreased transit time between plasma and water phases results in a decreased radical species recombination rate capable of offsetting the proposed decrease in hydroxyl concentration in the plasma region.

In addition to arguing that increasing hydroxyl concentration in the plasma region should increase the oxidizing nature of the discharge and subsequently increase nitrate concentrations, a stoichiometric outlook suggests that introducing another source of elemental oxygen increases the ratio of oxygen to nitrogen in the discharge, allowing greater formation of high O:N ratio species like NO$_3^-$.  This theory could be explored more in future experiments with varying feed ratios of N$_2$ and O$_2$.

\begin{figure}[htbp]
  \centering
  \includegraphics{H2O2_measurement.png}
  \caption{Hydrogen peroxide concentration in solution as a function of plasma power}
  \label{fig:H2O2}
\end{figure}

In order to address the last variable considered in the study, the effect of basic aqueous species on nitrogen ion concentrations, it is worthwhile to summarize some of the potentially important reaction mechanisms involving reactive nitrogen and oxygen species in solution.   Aside from nitrite and nitrate ions, hydrogen peroxide is known to be a prevailing species in solution following plasma treatment \cite{traylor2011long}; this is confirmed by colorimetric analysis in \cref{fig:H2O2}.  Moreover, volatile NO$_x$ species like NO and NO$_2$ may also be present and may be responsible for the observation in \cite{traylor2011long} of a spectroscopic peak at 262 nm when samples are sealed; when samples are left unsealed, the 262 nm peak is not observed.  Relevant redox reactions involving these species are taken from \cite{brisset2012peroxynitrite}  and \cite{greenwood1984chemistry} and presented in Table 7.

\begin{table}[htpb]
  \begin{center}
    \begin{tabular}{|l |c |}
      \hline
      \textbf{Reaction Description} & \textbf{Reaction reference \#} \\\hline
      $4NO + O_2 + 2H_2O \rightarrow 4H^+ + 4NO_2^-$ & 1 \\\hline
      $H_2O_2 + NO_2^- \rightarrow ONOO^- + H_2O$ & 2 \\\hline
      $ONOOH \rightarrow H^+ + NO_3^-$ & 3 \\\hline
      $3HNO_2 \rightarrow 2NO + NO_3^- + H^+ + H_2O$ & 4 \\\hline
      $NO + NO_2 + 2A^- + H_2O \rightarrow 2NO_2^- + 2HA$ & 5 \\\hline
      $2NO + O_2 \rightarrow NO_2$ & 6 \\\hline
      $3NO_2 + H_2O \rightarrow 2H^+ + 2NO_3^- + NO$ & 7 \\\hline
      $4NO_2\,(or\,2N_2O_4) + O_2 + 2H_2O \rightarrow 4HNO_3$ & 8 \\\hline
    \end{tabular}
  \end{center}
  \caption{Important reactions between nitrogen and oxygen species which may occur in the aqueous phase}
  \label{tab:reactions}
\end{table}

References \cite{greenwood1984chemistry} and \cite{Lukes2014b} illustrate that peroxynitrous acid (ONOOH) is formed as an unstable intermediate during the oxidation of acidified aqueous solutions of nitrites to nitrates using H$_2$O$_2$, and that such solutions are more highly oxidizing than either H$_2$O$_2$ or HNO$_2$ alone.  Because the conditions of the former statement are satisfied in PAW, it is reasonable to assume that peroxynitrous acid is the intermediate species between nitrite and nitrate as hypothesized in \cite{traylor2011long}.  Moreover, the much greater efficacy of PAW compared to a control mixture of nitric acid and hydrogen peroxide for degrading bacteria \cite{burlica2010bacteria} further suggests the presence of a reactive oxidizing species like peroxynitrous acid.

Applying the equations in \cref{tab:reactions} to the investigation of basic species effects on nitrogen ion formation provides some insight into observed trends in \cref{fig:nitrogen_vs_time_delay}, where the nitrite and nitrate molar concentrations in PAW as a function of time delay between plasma exposure and addition of sodium bicarbonate are shown.  The +3 oxidation state of nitrogen in water, e.g. nitrite/nitrous acid, is unstable at acidic pH.  Following plasma exposure, PAW is acidic and reaction (4) in \cref{tab:reactions} will occur as long as the solution is acidic.  Subsequently, for long time delays between exposure and base addition, the solution has time to convert nearly all aqueous nitrogen species into nitrate.  If base is added immediately following exposure more nitrite will be preserved in solution.  Moreover, if base is added while +2 and +4 oxidation state nitrogen is present in solution, e.g. species such as NO and NO$_2$, reaction (5) may occur.  It is conceivable that reaction (5) is responsible for the nitrite trends witnessed in \cref{tab:bicarb}.  Tap water contains more basic species than distilled water which theoretically contains none other than a 10-7 molar concentration of hydroxide.  Subsequently, tap water contains more A- species that are capable of reacting with NO and NO$_2$ to form nitrite.  Moreover, if a large quantity of additional base is added to solution before plasma exposure, the amount of A- available for reaction (5) increases significantly, leading perhaps to the comparatively large concentration of nitrite observed in sample 2 in Table 6.  This result is not observed to the same degree in the distilled water sample, sample 5, but some effect is still present.  Relative to solutions that received no additions, the increased presence of nitrite following post-treatment basic additions could be a combination of both reaction (4) and (5) effects, with (5) occurring when base is added quickly enough that NO and NO$_2$ are still dissolved in solution.

Much of the theory suggested above needs to be validated by further experiment and by computational models.  Models should include the relevant chemical reactions shown in \cite{moussa2005acidity}, \cite{brisset2012peroxynitrite}, and \cite{greenwood1984chemistry} and must be coupled to reactions and mass transfer from the plasma phase. With the construction of the models in \cref{chap:basic_science} and the flexbility of the code in \cref{chap:zapdos}, exploration of solution chemistry and the theories presented here are well within reach and are on the agenda for future research.

\section{Summary}

\Cref{chap:expt_opt} describes our experimental designs for investigating plasma-liquid interactions. \Cref{sec:base} describes the base configuration where the 162 MHz plasma source is pointed downward into a reservoir of water. \Cref{sec:spray} examines trying to increase the surface area of plasma-liquid interaction by directly introducing water droplets into the plasma dischare. In \cref{sec:electrodes} we discuss the electrodes placed on the end of the VHF source's inner conductor and their tendency for plasma erosion. To increase fluxes of charged plasma species to the water surface and to alleviate electrode damage, we introduce in \cref{sec:water_electrodes} an experimental configuration in which the source is pointed upwards and water is pumped through the inner conductor to form a liquid layer on top of the powered electrode. Finally, in \cref{sec:aq_chem} we measure different aqueous specie concentrations as a function of different system variables and speculate on the observed trends. The need to extend the models presented in \cref{chap:basic_science} to confirm some of the hypotheses in \cref{sec:aq_chem} is noted. Having built and characterized these experimental configurations, it is worthwhile to explore some of the applications of plasma-liquid systems. In \cref{chap:applications} we explore a couple of these applications, including fertilization of plants using plasma activated water and degradation of persistent aqueous contaminants.
