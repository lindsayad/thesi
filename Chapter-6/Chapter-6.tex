\chapter{CONCLUSION \& FUTURE WORK}
\label{chap:conclusion}

\section{Work to Date}

This dissertation includes both modelling and experimental studies of plasma-liquid systems. The keystones of the plasma-liquid research are the modelling chapters, \cref{chap:basic_science,chap:zapdos}. \Cref{chap:basic_science} addresses a couple of fundamental questions of the plasma-liquid research community. \Cref{sec:plasfree_model} considers the role of convective fluid flow on transport processes in plasma-liquid systems. Through evaporative cooling, convection creates a significant difference in bulk temperatures (roughly 10 K) between gas and liquid phases. Convection also significantly increases mass transfer rates from gas to liquid for hydrophobic species like NO. Additionally, penetration of reactive species like OH and surface ONOOH formed from reaction of OH and NO$_2$ into solution is limited to a few tens of microns. This suggests that reactivity at a substrate covered by an aqueous layer is likely due to longer lived species like NO$_2^-$ and H$_2$O$_2$ capable of reacting to create OH and NO$_2$ radicals through a peroxynitrous acid intermediate.

\Cref{sec:plasliq} describes a fully coupled discharge-liquid model in which the effect of varying interfacial parameters like the electron surface loss coefficient is considered. It is found that over a range of interfacial parameters, the near-interface gas electron density can vary by four orders of magnitude. The interfacial electron energy is similarly a sensitive function of loss coefficients. This motivates finer scale computational efforts like Molecular Dynamics simulations or novel experimental techniques to probe the near-interface plasma dynamics that are capable of accurately determining the interfacial cofficients required for fluid modelling.

In \cref{chap:zapdos}, we present the code Zapdos we developed for simulating plasma discharges in contact with liquids. Included in the chapter are descriptions of the kernels used to re-create plasma-liquid governing equations, auxiliary kernels for computing important system variables, materials used to describe features of the gas and liquid phases, boundary conditions describing the interactions of our domains with the environment, and interfacial kernels for connecting the physics in the plasma and liquid domains. We also describe the changes made to the MOOSE framework allowing coupling of the gas and liquid phases.

Experimental optimization and characterization of experimental plasma-liquid systems are the subjects of \cref{chap:expt_opt}. Several configurations based around our 162 MHz VHF source are described. The most successful of these points the VHF source upward; water is then pumped through the center of the inner conductor and an approximately milimeter thick layer of water is formed on top of the powered electrode. Because no powered surfaces are exposed to the plasma in the latter configuration, the system can be operated at much higher powers than normal, up to 1155 W. Additionally, OH is an abundant plasma specie as indicated by broadband absorption spectroscopy. The final section of \cref{chap:expt_opt} explores the aqueous chemistry generated by contact with the plasma. The chemistry depends sensitively on the pre-existing solution chemistry; it shows a weaker dependence on parameters like power and gas flow rate. NO$_3^-$ and H$_2$O$_2$ are produced ubiquitously.

\Cref{chap:applications} explores a couple of applications of plasma-liquids. In a collaborative experiment with the horticulture department, plasma treated water was used to water tomatoes, marigolds, and radishes, and its performance was compared to tap-water controls. The plasma water treated plants grew significantly larger than controls because of the >100 ppm nitrate concentration generated in solution by the plasma. Plasma was also shown to be effective at remediating persistent aqueous pollutants like 1,4-dioxane and perfluorooctanesulfonate.

\section{Future Work}

\begin{figure}[htbp]
  \centering
  \includegraphics[width=\textwidth]{go_discharge_spreading.png}
  \caption{Figure taken from \cite{rumbach2015solvation} showing the spreading of a DC discharge as solution conductivity is decreased.}
  \label{fig:go_spreading}
\end{figure}

There is potential for numerous fascinating projects stemming from the work presented in this dissertation. First on the agenda is extending the model in \cref{sec:plasliq} to multiple dimensions. Using a 2D axisymmetric model, we hope to reproduce the experimentally observed behavior shown in \cref{fig:go_spreading}: increased spreading of a DC discharge over a water surface as solution conductivity is decreased. Once the discharge model is vetted in multiple dimensions, it can be combined with the model in \cref{sec:plasfree_model} to provide a comprehensive description of the physiochemical phenomena present for point-to-plane DC discharges. Addition to Zapdos of the momentum and heat transport kernels needed for model coupling should completable within a day. The more challenging aspect will be defining consistent and numerically feasible boundary conditions for the potential in multiple dimensions.

We also plan to add electromagnetic field modelling capabilities to Zapdos. With EM capability we can begin to model the experimental systems described in \cref{chap:expt_opt}. The VHF plasma-liquid system presents a unique blend of physical and chemical phenomena, including EM fields, charged and neutral particle transport, fluid flow, heat transport, and gas and liquid chemistry. Zapdos is ideally suited for combining such rich dynamics. If such a comprehensive model of the VHF plasma-liquid system can be built, then it can be compared to some of the measurements made in \cref{sec:aq_chem}. If the model compares favorably to the measurements, then the model can be used to elucidate the mechanisms that lead to long-lived aqueous species like H$_2$O$_2$, NO$_2^-$, and NO$_3^-$. This could be beneficial for determining optimal plasma parameters for production of fertigation water. Additionally, reaction kinetics for dioxane and PFOS could be added to the model in an attempt to describe the behavior seen in \cref{sec:pollutant} and optimize the system for pollutant degradation.

Zapdos can also be extended in a couple of other ways. To accurately describe electron behavior in the cathode of DC discharges and perhaps at the plasma-liquid interface, a kinetic instead of fluid model is required. MOOSE is capable of defining independent variables in addition to the traditional three spatial coordinates and time; the developers of RATTLESNAKE, another MOOSE application, use an energy variable as well as two streaming variables. It is conceivable that a kinetic model could be developed for the sheath regions and self-consistently coupled to a fluid model in the bulk regions, all underneath the Zapdos umbrella. In addition to kinetic models in the sheath it may be possible to self-consistently incorporate atomistic or molecular dynamic simulations of the interface to accurately capture interfacial electron behavior. There are current efforts in the MOOSE community to couple SPPARKS, a kinetic Monte Carlo code, and LAMMPS, a molecular dynamics simulator, to MOOSE applications.

Although not directly related to plasma-liquids, other students in our group are looking to extend Zapdos to modelling of RF antennaes on structures like ITER as well as simulation of low-pressure plasmas related to semiconductor manufacture. Additionally a group in Chengdu, China is interested in using Zapdos to model CO$_2$ to CO conversion with microwave and RF plasmas. With a host of potential projects, the future of Zapdos appears bright!
