\chapter{Relevant Code Snippets}
\label{chap:code_snippets}

\section{Zapdos Input File}
\label{sec:zap_input}

Below is a Zapdos input file that is exemplary of the input files used to produce the simulation results in \cref{sec:plasliq}. For descriptions of many of the input blocks and the classes used in them, see \cref{sec:zapdos}.

\begin{lstlisting}[caption = Exemplary Zapdos input file, label = code:input_file]
dom0Scale=1e-3
dom1Scale=1e-7

[GlobalParams]
  offset = 20
  # offset = 0
  potential_units = kV
  # potential_units = V
[]

[Mesh]
  # type = GeneratedMesh
  # nx = 2
  # xmax = 1.1
  # dim = 1
  type = FileMesh
  file = 'liquidNew.msh'
[]

[MeshModifiers]
  [./interface]
    type = SideSetsBetweenSubdomains
    master_block = '0'
    paired_block = '1'
    new_boundary = 'master0_interface'
    # depends_on = 'box'
  [../]
  [./interface_again]
    type = SideSetsBetweenSubdomains
    master_block = '1'
    paired_block = '0'
    new_boundary = 'master1_interface'
    # depends_on = 'box'
  [../]
  [./left]
    type = SideSetsFromNormals
    normals = '-1 0 0'
    new_boundary = 'left'
  [../]
  [./right]
    type = SideSetsFromNormals
    normals = '1 0 0'
    new_boundary = 'right'
  [../]
  # [./box]
  #   type = SubdomainBoundingBox
  #   bottom_left = '0.55 0 0'
  #   top_right = '1.1 1. 0'
  #   block_id = 1
  # [../]
[]

[Problem]
  type = FEProblem
  # kernel_coverage_check = false
[]

[Preconditioning]
  [./smp]
    type = SMP
    full = true
  [../]
[]

[Executioner]
  type = Transient
  end_time = 1e-1
  # end_time = 10
  petsc_options = '-snes_converged_reason -snes_linesearch_monitor'
  # petsc_options = '-snes_test_display'
  solve_type = NEWTON
  petsc_options_iname = '-pc_type -pc_factor_shift_type -pc_factor_shift_amount -ksp_type -snes_linesearch_minlambda'
  petsc_options_value = 'lu NONZERO 1.e-10 preonly 1e-3'
  # petsc_options_iname = '-snes_type'
  # petsc_options_value = 'test'
 nl_rel_tol = 1e-4
 nl_abs_tol = 5.4e-5
  dtmin = 1e-12
  [./TimeStepper]
    type = IterationAdaptiveDT
    cutback_factor = 0.4
    dt = 1e-9
    # dt = 1.1
    growth_factor = 1.2
   optimal_iterations = 15
  [../]
[]

[Outputs]
  print_perf_log = true
  print_linear_residuals = false
  [./out]
    type = Exodus
  [../]
[]

[Debug]
  show_var_residual_norms = true
[]

[UserObjects]
  [./data_provider]
    type = ProvideMobility
    electrode_area = 5.02e-7 # Formerly 3.14e-6
    ballast_resist = 1e6
    e = 1.6e-19
    # electrode_area = 1.1
    # ballast_resist = 1.1
    # e = 1.1
  [../]
[]

[Kernels]
  [./em_time_deriv]
    type = ElectronTimeDerivative
    variable = em
    block = 0
  [../]
  [./em_advection]
    type = EFieldAdvectionElectrons
    variable = em
    potential = potential
    mean_en = mean_en
    block = 0
    position_units = ${dom0Scale}
  [../]
  [./em_diffusion]
    type = CoeffDiffusionElectrons
    variable = em
    mean_en = mean_en
    block = 0
    position_units = ${dom0Scale}
  [../]
  [./em_ionization]
    type = ElectronsFromIonization
    variable = em
    potential = potential
    mean_en = mean_en
    block = 0
    position_units = ${dom0Scale}
  [../]
  [./em_log_stabilization]
    type = LogStabilizationMoles
    variable = em
    block = 0
  [../]
  # [./em_advection_stabilization]
  #   type = EFieldArtDiff
  #   variable = em
  #   potential = potential
  # [../]

  [./emliq_time_deriv]
    type = ElectronTimeDerivative
    variable = emliq
    block = 1
  [../]
  [./emliq_advection]
    type = EFieldAdvection
    variable = emliq
    potential = potential
    block = 1
    position_units = ${dom1Scale}
  [../]
  [./emliq_diffusion]
    type = CoeffDiffusion
    variable = emliq
    block = 1
    position_units = ${dom1Scale}
  [../]
  [./emliq_reactant_first_order_rxn]
    type = ReactantFirstOrderRxn
    variable = emliq
    block = 1
  [../]
  [./emliq_water_bi_sink]
    type = ReactantAARxn
    variable = emliq
    block = 1
  [../]
  [./emliq_log_stabilization]
    type = LogStabilizationMoles
    variable = emliq
    block = 1
  [../]

  [./potential_diffusion_dom1]
    type = CoeffDiffusionLin
    variable = potential
    block = 0
    position_units = ${dom0Scale}
  [../]
  [./potential_diffusion_dom2]
    type = CoeffDiffusionLin
    variable = potential
    block = 1
    position_units = ${dom1Scale}
  [../]
  [./Arp_charge_source]
    type = ChargeSourceMoles_KV
    variable = potential
    charged = Arp
    block = 0
  [../]
  [./em_charge_source]
    type = ChargeSourceMoles_KV
    variable = potential
    charged = em
    block = 0
  [../]
  [./emliq_charge_source]
    type = ChargeSourceMoles_KV
    variable = potential
    charged = emliq
    block = 1
  [../]
  [./OHm_charge_source]
    type = ChargeSourceMoles_KV
    variable = potential
    charged = OHm
    block = 1
  [../]

  [./Arp_time_deriv]
    type = ElectronTimeDerivative
    variable = Arp
    block = 0
  [../]
  [./Arp_advection]
    type = EFieldAdvection
    variable = Arp
    potential = potential
    position_units = ${dom0Scale}
    block = 0
  [../]
  [./Arp_diffusion]
    type = CoeffDiffusion
    variable = Arp
    block = 0
    position_units = ${dom0Scale}
  [../]
  [./Arp_ionization]
    type = IonsFromIonization
    variable = Arp
    potential = potential
    em = em
    mean_en = mean_en
    block = 0
    position_units = ${dom0Scale}
  [../]
  [./Arp_log_stabilization]
    type = LogStabilizationMoles
    variable = Arp
    block = 0
  [../]
  # [./Arp_advection_stabilization]
  #   type = EFieldArtDiff
  #   variable = Arp
  #   potential = potential
  # [../]

  [./OHm_time_deriv]
    type = ElectronTimeDerivative
    variable = OHm
    block = 1
  [../]
  [./OHm_advection]
    type = EFieldAdvection
    variable = OHm
    potential = potential
    block = 1
    position_units = ${dom1Scale}
  [../]
  [./OHm_diffusion]
    type = CoeffDiffusion
    variable = OHm
    block = 1
    position_units = ${dom1Scale}
  [../]
  [./OHm_log_stabilization]
    type = LogStabilizationMoles
    variable = OHm
    block = 1
  [../]
  # [./OHm_advection_stabilization]
  #   type = EFieldArtDiff
  #   variable = OHm
  #   potential = potential
  #   block = 1
  # [../]
  [./OHm_product_first_order_rxn]
    type = ProductFirstOrderRxn
    variable = OHm
    v = emliq
    block = 1
  [../]
  [./OHm_product_aabb_rxn]
    type = ProductAABBRxn
    variable = OHm
    v = emliq
    block = 1
  [../]

  [./mean_en_time_deriv]
    type = ElectronTimeDerivative
    variable = mean_en
    block = 0
  [../]
  [./mean_en_advection]
    type = EFieldAdvectionEnergy
    variable = mean_en
    potential = potential
    em = em
    block = 0
    position_units = ${dom0Scale}
  [../]
  [./mean_en_diffusion]
    type = CoeffDiffusionEnergy
    variable = mean_en
    em = em
    block = 0
    position_units = ${dom0Scale}
  [../]
  [./mean_en_joule_heating]
    type = JouleHeating
    variable = mean_en
    potential = potential
    em = em
    block = 0
    position_units = ${dom0Scale}
  [../]
  [./mean_en_ionization]
    type = ElectronEnergyLossFromIonization
    variable = mean_en
    potential = potential
    em = em
    block = 0
    position_units = ${dom0Scale}
  [../]
  [./mean_en_elastic]
    type = ElectronEnergyLossFromElastic
    variable = mean_en
    potential = potential
    em = em
    block = 0
    position_units = ${dom0Scale}
  [../]
  [./mean_en_excitation]
    type = ElectronEnergyLossFromExcitation
    variable = mean_en
    potential = potential
    em = em
    block = 0
    position_units = ${dom0Scale}
  [../]
  [./mean_en_log_stabilization]
    type = LogStabilizationMoles
    variable = mean_en
    block = 0
    offset = 15
  [../]
  # [./mean_en_advection_stabilization]
  #   type = EFieldArtDiff
  #   variable = mean_en
  #   potential = potential
  # [../]
[]

[Variables]
  [./potential]
  [../]
  [./em]
    block = 0
  [../]
  [./emliq]
    block = 1
    # scaling = 1e-4
  [../]

  [./Arp]
    block = 0
  [../]

  [./mean_en]
    block = 0
    # scaling = 1e-1
  [../]

  [./OHm]
    block = 1
    # scaling = 1e-4
  [../]
[]

[AuxVariables]
  [./e_temp]
    block = 0
  [../]
  [./x]
    order = CONSTANT
    family = MONOMIAL
  [../]
  [./rho]
    order = CONSTANT
    family = MONOMIAL
    block = 0
  [../]
  [./rholiq]
    block = 1
    order = CONSTANT
    family = MONOMIAL
  [../]
  [./em_lin]
    order = CONSTANT
    family = MONOMIAL
    block = 0
  [../]
  [./emliq_lin]
    order = CONSTANT
    family = MONOMIAL
    block = 1
  [../]
  [./Arp_lin]
    order = CONSTANT
    family = MONOMIAL
    block = 0
  [../]
  [./OHm_lin]
    block = 1
    order = CONSTANT
    family = MONOMIAL
  [../]
  [./Efield]
    order = CONSTANT
    family = MONOMIAL
  [../]
  [./Current_em]
    order = CONSTANT
    family = MONOMIAL
    block = 0
  [../]
  [./Current_emliq]
    order = CONSTANT
    family = MONOMIAL
    block = 1
  [../]
  [./Current_Arp]
    order = CONSTANT
    family = MONOMIAL
    block = 0
  [../]
  [./Current_OHm]
    block = 1
    order = CONSTANT
    family = MONOMIAL
  [../]
  [./tot_gas_current]
    order = CONSTANT
    family = MONOMIAL
    block = 0
  [../]
  [./tot_liq_current]
    block = 1
    order = CONSTANT
    family = MONOMIAL
  [../]
  [./tot_flux_OHm]
    block = 1
    order = CONSTANT
    family = MONOMIAL
  [../]
  [./EFieldAdvAux_em]
    order = CONSTANT
    family = MONOMIAL
    block = 0
  [../]
  [./DiffusiveFlux_em]
    order = CONSTANT
    family = MONOMIAL
    block = 0
  [../]
  [./EFieldAdvAux_emliq]
    order = CONSTANT
    family = MONOMIAL
    block = 1
  [../]
  [./DiffusiveFlux_emliq]
    order = CONSTANT
    family = MONOMIAL
    block = 1
  [../]
  [./PowerDep_em]
   order = CONSTANT
   family = MONOMIAL
   block = 0
  [../]
  [./PowerDep_Arp]
   order = CONSTANT
   family = MONOMIAL
   block = 0
  [../]
  [./ProcRate_el]
   order = CONSTANT
   family = MONOMIAL
   block = 0
  [../]
  [./ProcRate_ex]
   order = CONSTANT
   family = MONOMIAL
   block = 0
  [../]
  [./ProcRate_iz]
   order = CONSTANT
   family = MONOMIAL
   block = 0
  [../]
[]

[AuxKernels]
  [./PowerDep_em]
    type = PowerDep
    density_log = em
    potential = potential
    art_diff = false
    potential_units = kV
    variable = PowerDep_em
  [../]
  [./PowerDep_Arp]
    type = PowerDep
    density_log = Arp
    potential = potential
    art_diff = false
    potential_units = kV
    variable = PowerDep_Arp
  [../]
  [./ProcRate_el]
    type = ProcRate
    em = em
    potential = potential
    proc = el
    variable = ProcRate_el
  [../]
  [./ProcRate_ex]
    type = ProcRate
    em = em
    potential = potential
    proc = ex
    variable = ProcRate_ex
  [../]
  [./ProcRate_iz]
    type = ProcRate
    em = em
    potential = potential
    proc = iz
    variable = ProcRate_iz
  [../]
  [./e_temp]
    type = ElectronTemperature
    variable = e_temp
    electron_density = em
    mean_en = mean_en
    block = 0
  [../]
  [./x]
    type = Position
    variable = x
  [../]
  [./rho]
    type = ParsedAux
    variable = rho
    args = 'em_lin Arp_lin'
    function = 'Arp_lin - em_lin'
    execute_on = 'timestep_end'
    block = 0
  [../]
  [./rholiq]
    type = ParsedAux
    variable = rholiq
    args = 'emliq_lin OHm_lin' # H3Op_lin OHm_lin'
    function = '-emliq_lin - OHm_lin' # 'H3Op_lin - em_lin - OHm_lin'
    execute_on = 'timestep_end'
    block = 1
  [../]
  [./tot_gas_current]
    type = ParsedAux
    variable = tot_gas_current
    args = 'Current_em Current_Arp'
    function = 'Current_em + Current_Arp'
    execute_on = 'timestep_end'
    block = 0
  [../]
  [./tot_liq_current]
    type = ParsedAux
    variable = tot_liq_current
    args = 'Current_emliq Current_OHm' # Current_H3Op Current_OHm'
    function = 'Current_emliq + Current_OHm' # + Current_H3Op + Current_OHm'
    execute_on = 'timestep_end'
    block = 1
  [../]
  [./em_lin]
    type = Density
    variable = em_lin
    density_log = em
    block = 0
  [../]
  [./emliq_lin]
    type = Density
    variable = emliq_lin
    density_log = emliq
    block = 1
  [../]
  [./Arp_lin]
    type = Density
    variable = Arp_lin
    density_log = Arp
    block = 0
  [../]
  [./OHm_lin]
    type = Density
    variable = OHm_lin
    density_log = OHm
    block = 1
  [../]
  [./Efield]
    type = Efield
    potential = potential
    variable = Efield
  [../]
  [./Current_em]
    type = Current
    potential = potential
    density_log = em
    variable = Current_em
    art_diff = false
    block = 0
    position_units = ${dom0Scale}
  [../]
  [./Current_emliq]
    type = Current
    potential = potential
    density_log = emliq
    variable = Current_emliq
    art_diff = false
    block = 1
    position_units = ${dom1Scale}
  [../]
  [./Current_Arp]
    type = Current
    potential = potential
    density_log = Arp
    variable = Current_Arp
    art_diff = false
    block = 0
    position_units = ${dom0Scale}
  [../]
  [./Current_OHm]
    block = 1
    type = Current
    potential = potential
    density_log = OHm
    variable = Current_OHm
    art_diff = false
    position_units = ${dom1Scale}
  [../]
  [./tot_flux_OHm]
    block = 1
    type = TotalFlux
    potential = potential
    density_log = OHm
    variable = tot_flux_OHm
  [../]
  [./EFieldAdvAux_em]
    type = EFieldAdvAux
    potential = potential
    density_log = em
    variable = EFieldAdvAux_em
    block = 0
  [../]
  [./DiffusiveFlux_em]
    type = DiffusiveFlux
    density_log = em
    variable = DiffusiveFlux_em
    block = 0
  [../]
  [./EFieldAdvAux_emliq]
    type = EFieldAdvAux
    potential = potential
    density_log = emliq
    variable = EFieldAdvAux_emliq
    block = 1
  [../]
  [./DiffusiveFlux_emliq]
    type = DiffusiveFlux
    density_log = emliq
    variable = DiffusiveFlux_emliq
    block = 1
  [../]
[]

[InterfaceKernels]
  [./em_advection]
    type = InterfaceAdvection
    mean_en_neighbor = mean_en
    potential_neighbor = potential
    neighbor_var = em
    variable = emliq
    boundary = master1_interface
    position_units = ${dom1Scale}
    neighbor_position_units = ${dom0Scale}
  [../]
  [./em_diffusion]
    type = InterfaceLogDiffusionElectrons
    mean_en_neighbor = mean_en
    neighbor_var = em
    variable = emliq
    boundary = master1_interface
    position_units = ${dom1Scale}
    neighbor_position_units = ${dom0Scale}
  [../]
[]

[BCs]
  [./potential_left]
    type = NeumannCircuitVoltageMoles_KV
    variable = potential
    boundary = left
    function = potential_bc_func
    ip = Arp
    data_provider = data_provider
    em = em
    mean_en = mean_en
    r = 0
    position_units = ${dom0Scale}
  [../]
  [./potential_dirichlet_right]
    type = DirichletBC
    variable = potential
    boundary = right
    value = 0
  [../]
  [./em_physical_right]
    type = HagelaarElectronBC
    variable = em
    boundary = 'master0_interface'
    potential = potential
    ip = Arp
    mean_en = mean_en
    r = 0
    position_units = ${dom0Scale}
  [../]
  # [./em_physical_right]
  #   type = MatchedValueLogBC
  #   variable = em
  #   boundary = 'master0_interface'
  #   v = emliq
  #   H = 1e7
  # [../]
  [./Arp_physical_right]
    type = HagelaarIonBC
    variable = Arp
    boundary = 'master0_interface'
    potential = potential
    r = 0
    position_units = ${dom0Scale}
  [../]
  [./mean_en_physical_right]
    type = HagelaarEnergyBC
    variable = mean_en
    boundary = 'master0_interface'
    potential = potential
    em = em
    ip = Arp
    r = 0
    position_units = ${dom0Scale}
  [../]
  [./em_physical_left]
    type = HagelaarElectronBC
    variable = em
    boundary = 'left'
    potential = potential
    ip = Arp
    mean_en = mean_en
    r = 0
    position_units = ${dom0Scale}
  [../]
  [./Arp_physical_left]
    type = HagelaarIonBC
    variable = Arp
    boundary = 'left'
    potential = potential
    r = 0
    position_units = ${dom0Scale}
  [../]
  [./mean_en_physical_left]
    type = HagelaarEnergyBC
    variable = mean_en
    boundary = 'left'
    potential = potential
    em = em
    ip = Arp
    r = 0
    position_units = ${dom0Scale}
  [../]
  [./emliq_right]
    type = DCIonBC
    variable = emliq
    boundary = right
    potential = potential
    position_units = ${dom1Scale}
  [../]
  [./OHm_physical]
    type = DCIonBC
    variable = OHm
    boundary = 'right'
    potential = potential
    position_units = ${dom1Scale}
  [../]
[]

[ICs]
  [./em_ic]
    type = ConstantIC
    variable = em
    value = -26
    block = 0
  [../]
  [./emliq_ic]
    type = ConstantIC
    variable = emliq
    value = -21
    block = 1
  [../]
  [./Arp_ic]
    type = ConstantIC
    variable = Arp
    value = -26
    block = 0
  [../]
  [./mean_en_ic]
    type = ConstantIC
    variable = mean_en
    value = -25
    block = 0
  [../]
  # [./potential_ic]
  #   type = ConstantIC
  #   variable = potential
  #   value = 0
  # [../]
  [./potential_ic]
    type = FunctionIC
    variable = potential
    function = potential_ic_func
  [../]
  [./OHm_ic]
    type = ConstantIC
    variable = OHm
    value = -15.6
    block = 1
  [../]
  # [./em_ic]
  #   type = RandomIC
  #   variable = em
  #   block = 0
  # [../]
  # [./emliq_ic]
  #   type = RandomIC
  #   variable = emliq
  #   block = 1
  # [../]
  # [./Arp_ic]
  #   type = RandomIC
  #   variable = Arp
  #   block = 0
  # [../]
  # [./mean_en_ic]
  #   type = RandomIC
  #   variable = mean_en
  #   block = 0
  # [../]
  # [./potential_ic]
  #   type = RandomIC
  #   variable = potential
  # [../]
  # [./OHm_ic]
  #   type = RandomIC
  #   variable = OHm
  #   block = 1
  # [../]
[]

[Functions]
  [./potential_bc_func]
    type = ParsedFunction
    # value = '1.25*tanh(1e6*t)'
    value = 1.25
  [../]
  [./potential_ic_func]
    type = ParsedFunction
    value = '-1.25 * (1.0001e-3 - x)'
  [../]
[]

[Materials]
  [./gas_block]
    type = Gas
    interp_trans_coeffs = true
    interp_elastic_coeff = true
    em = em
    potential = potential
    ip = Arp
    mean_en = mean_en
    block = 0
 [../]
 [./water_block]
   type = Water
   block = 1
   potential = potential
 [../]
 # [./jac]
 #   type = JacMat
 #   mean_en = mean_en
 #   em = em
 #   emliq = emliq
 #   block = '0 1'
 #  [../]
[]
\end{lstlisting}

\section{Python Methods}
\label{sec:python}

A couple of useful python methods we developed are described below. Interested readers can navigate to the python directory of \cite{alexProgramming} for more python scripts (at various levels of code maturity).

\subsection{load\_data method}
\label{sec:load_data}

The load\_data method shown in \cref{code:load_data} handles a raw Exodus data file output by Zapdos. It uses reader and writer modules from Paraview's \cite{paraview} python interface to write solution variable data from the last simulation time step to a CSV file. Then numpy is used to load node and element data into separate arrays for later processing. To run the load\_data method, the user specifies the following arguments, all of which are lists and must have the same size (elements correspond to a particular job):

\begin{itemize}
\label{list:load_dat_args}
  \item short\_names (type list of strings): The strings passed in this list should summarize the simulation jobs. They will be used as the label strings later when solution variables like electron density, potential, etc. are plotted
  \item labels (type list of strings): List of colors that corresponds to the list of jobs that will be used in coloring variable plots, e.g. labels = ['blue','red'] would mean that job1 plots would be blue and job2 plots would be red
  \item mesh\_struct (type list of strings): The elements in this list must either equal 'scaled' or 'physical'. 'Physical' means that the simulation did not use a scaled mesh.
  \item styles (type list of strings): Options for elements in this list include 'solid', 'dashed', 'dashdot', etc. A job with a style of 'dashed' will have dashed line plots.
\end{itemize}

\begin{lstlisting}[language = Python, caption = Python method for turning raw Exodus file data into numpy arrays using Paraview's \cite{paraview} python interface, label = code:load_data]
def load_data(short_names, labels, mesh_struct, styles):
    global job_names, name_dict, cellGasData, cellLiquidData, pointGasData, pointLiquidData, label_dict, style_dict, mesh_dict
    data = OrderedDict()
    cellData = OrderedDict()

    name_dict = {x:y for x,y in zip(job_names, short_names)}
    label_dict = {x:y for x,y in zip(job_names, labels)}
    style_dict = {x:y for x,y in zip(job_names, styles)}
    mesh_dict = {x:y for x,y in zip(job_names, mesh_struct)}

    index = 0
    GasElemMax = 0
    path = "/home/lindsayad/gdrive/MooseOutput/"
    for job in job_names:
        file_sans_ext = path + job + "_gold_out"
        inp = file_sans_ext + ".e"
        out = file_sans_ext + ".csv"

        reader = ExodusIIReader(FileName=inp)
        tsteps = reader.TimestepValues
        writer = CreateWriter(out, reader)
        writer.Precision = 16
        writer.UpdatePipeline(time=tsteps[len(tsteps)-1])
        del writer

        for i in range(2,6):
            os.remove(file_sans_ext + str(i) + ".csv")

        new_inp0 = file_sans_ext + "0.csv"
        data[job] = np.genfromtxt(new_inp0,delimiter=',', names=True)
        pointGasData[job] = data[job]

        # Use for coupled gas-liquid simulations
        new_inp1 = file_sans_ext + "1.csv"
        data1 = np.genfromtxt(new_inp1, delimiter=',', names=True)
        pointLiquidData[job] = data1
        data[job] = np.concatenate((data[job],data1), axis=0)

        writer = CreateWriter(out, reader)
        writer.FieldAssociation = "Cells"
        writer.Precision = 16
        writer.UpdatePipeline(time=tsteps[len(tsteps)-1])
        del writer

        for i in range(2,6):
            os.remove(file_sans_ext + str(i) + ".csv")

        new_inp0 = file_sans_ext + "0.csv"
        cellData[job] = np.genfromtxt(new_inp0,delimiter=',', names=True)
        cellGasData[job] = cellData[job]
        if index == 0:
            GasElemMax = np.amax(cellData[job]['GlobalElementId'])

        # Use for coupled gas-liquid simulations
        new_inp1 = file_sans_ext + "1.csv"
        data1 = np.genfromtxt(new_inp1, delimiter=',', names=True)
        cellData[job] = np.concatenate((cellData[job],data1), axis=0)
        cellLiquidData[job] = data1
\end{lstlisting}

\subsection{Plotting methods}
\label{sec:plotting_methods}

The following two methods are for plotting elemental and nodal variables respectively. The user specifies the following parameters:

\begin{itemize}
\label{list:plot_args}
  \item save (type bool): Whether to output an eps figure
  \item variable (type str): The variable to be plotted, e.g. the electron density, potential, electron temperature, etc.
  \item pos\_scaling (type float): The amount of scaling of the mesh. This option will be deprecated now that the Position AuxKernel has been updated
  \item ylabel (type str): Label to apply to the y-axis
  \item tight\_plot (type bool): Whether to tighten the figure area. In general this should be True; True generally prevents cutting off of x- and y-axis labels that may occur if False
  \item xticks (type list of floats): Where to apply tick marks on the axes
  \item xticklabels (type list of strings): List of tick labels being applied to the ticks passed to xticks
  \item xlabel (type str): Label to apply to the x-axis
  \item xmin (type float): Optional argument. Specifies x-axis minimum
  \item xmax (type float): Optional argument. Specifies x-axis maximum
  \item ymin (type float): Optional argument. Specifies y-axis minimum
  \item ymax (type float): Optional argument. Specifies y-axis maximum
  \item yscale (type str): Optional argument. Specifies the format of the y-axis. Can pass 'log' to specify logarithmic scaling
  \item save\_string (type str): Optional argument. If save=True, the user should specify a string here to create unique name for the figure saved. Otherwise the string 'dummy' will be used.
\end{itemize}

The elemental variable plotting method is given in \cref{code:elem_plot}.

\begin{lstlisting}[language=Python, caption = Generic elemental variable plotting method. Argument description given in \cref{list:plot_args}, label = code:elem_plot]
def cell_gas_generic(save, variable, pos_scaling, ylabel, tight_plot, xticks, xticklabels, xlabel, xmin=None, xmax=None, ymin=None, ymax=None, yscale=None, save_string="dummy"):
    fig = plt.figure()
    ax1 = plt.subplot(111)
    for job in job_names:
        if mesh_dict[job] == "phys":
            ax1.plot(cellGasData[job]['x'], cellGasData[job][variable], color = label_dict[job], linestyle = style_dict[job], label = name_dict[job], linewidth=2)
        elif mesh_dict[job] == "scaled":
            ax1.plot(cellGasData[job]['x'] / pos_scaling, cellGasData[job][variable], color = label_dict[job], linestyle = style_dict[job], label = name_dict[job], linewidth=2)
    ax1.legend(loc='best', fontsize = 16)
    ax1.set_xticks(xticks)
    ax1.set_xticklabels(xticklabels)
    ax1.set_xlabel(xlabel)
    ax1.set_ylabel(ylabel)
    if xmin is not None:
        ax1.set_xlim(left=xmin)
    if xmax is not None:
        ax1.set_xlim(right=xmax)
    if ymin is not None:
        ax1.set_ylim(bottom=ymin)
    if ymax is not None:
        ax1.set_ylim(top=ymax)
    if yscale is not None:
        ax1.set_yscale(yscale)
    if tight_plot:
        fig.tight_layout()
    if save:
        fig.savefig('/home/lindsayad/Pictures/' + save_string + '_' + variable + '.eps', format='eps')
    plt.show()
\end{lstlisting}

The nodal variable plotting method is given in \cref{code:nodal_plot}.

\begin{lstlisting}[language=Python, caption = Generic nodal variable plotting method. Argument description given in \cref{list:plot_args}, label = code:nodal_plot]
def point_gas_generic(save, variable, pos_scaling, ylabel, tight_plot, xticks, xticklabels, xlabel, xmin=None, xmax=None, ymin=None, ymax=None, yscale=None, save_string="dummy"):
    fig = plt.figure()
    ax1 = plt.subplot(111)
    for job in job_names:
        if mesh_dict[job] == "phys":
            ax1.plot(pointGasData[job]['Points0'], pointGasData[job][variable], color = label_dict[job], linestyle = style_dict[job], label = name_dict[job], linewidth=2)
        elif mesh_dict[job] == "scaled":
            ax1.plot(pointGasData[job]['Points0'] / pos_scaling, pointGasData[job][variable], color = label_dict[job], linestyle = style_dict[job], label = name_dict[job], linewidth=2)
    ax1.legend(loc='best', fontsize = 16)
    ax1.set_xticks(xticks)
    ax1.set_xticklabels(xticklabels)
    ax1.set_xlabel(xlabel)
    ax1.set_ylabel(ylabel)
    if xmin is not None:
        ax1.set_xlim(left=xmin)
    if xmax is not None:
        ax1.set_xlim(right=xmax)
    if ymin is not None:
        ax1.set_ylim(bottom=ymin)
    if ymax is not None:
        ax1.set_ylim(top=ymax)
    if yscale is not None:
        ax1.set_yscale(yscale)
    if tight_plot:
        fig.tight_layout()
    if save:
        fig.savefig('/home/lindsayad/Pictures/' + save_string + '_' + variable + '.eps', format='eps')
    plt.show()
\end{lstlisting}
