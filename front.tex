%% ------------------------------ Abstract ---------------------------------- %%
\begin{abstract}

Plasma-liquids have exciting applications to several important socioeconomic areas, including agriculture, water treatment, and medicine. In order to realize their application potential, the basic physical and chemical phenomena of plasma-liquid systems must be better understood. Additionally, system designs must be optimized in order to maximize fluxes of critical plasma species to the liquid phase. To address these needs, we have performed both comprehensive modelling and experimental work. To date, models of plasma-liquids have focused on configurations where diffusion is the dominant transport process in both gas and liquid phases. However, convection plays a key role in many popular plasma source designs, including jets, corona discharges, and torches. In this dissertation, we model momentum, heat, and neutral species mass transfer in a convection-dominated system based on a corona discharge. We show that evaporative cooling produced by gas-phase convection can lead to a significant difference between gas and liquid phase bulk temperatures. Additionally, convection induced in the liquid phase by the gas phase flow substantially increases interfacial mass transfer of hydrophobic species like NO and NO$_2$. Finally, liquid kinetic modelling suggests that concentrations of highly reactive species like OH and ONOOH are several orders of magnitude higher at the interface than in the solution bulk.

Subsequent modelling has focused on coupling discharge physics with species transport at and through the interface. An assumption commonly seen in the literature is that interfacial loss coefficients of charged species like electrons are equal to unity. However, there is no experimental evidence to either deny or support this assumption. Without knowing the true interfacial behavior of electrons, we have explored the effects on key plasma-liquid variables of varying interfacial parameters like the electron and energy surface loss coefficients. Within a reasonable range for these parameters, we have demonstrated that the electron density on the gas phase side of the interface can vary by orders of magnitude. Significant effects can also be seen on the gas phase interfacial electron energy. Electron density and energy will play important roles in determining gas phase chemistry in more complex future models; this will in turn feed back into the liquid phase chemistry. In order to remove this uncertainty in interfacial behavior, we recommend finer scale atomistic or molecular dynamics simulations. Efficient coupling of the highly non-linear discharge physics equations to liquid transport required creation of a new simulation code named Zapdos, built on top of the MOOSE framework. The operation and capabilities of the code are described in this work. Moreover, changes made to the MOOSE framework allowing coupling of physics across subdomain boundaries, necessary for plasma-liquid coupling, are also detailed.

In the latter half of this work, we investigate experimental optimization and characterization of plasma-liquid interactions surrounding a unique very high frequency (VHF) plasma discharge. Several geometric configurations are considered. In the most promising set-up, the discharge is pointed upwards and water is pumped through the source's inner conductor until it forms a milimeter thick water layer on top of the powered electrode. This maximizes the amount of charged and neutral species flux received by the aqueous phase as well as the amount of water vapor created in the gas phase. Additionally, the configuration eliminates electrode damage by providing an infinitely renewable liquid surface layer. The presence of large amounts of water vapor and OH radicals is confirmed by optical emission and broadband absorption spectroscopy. Characterization of liquid phase species like NO$_3^-$, NO$_2^-$, and H$_2$O$_2$ is carried out through ion chromatography (IC) and colorimetric measurements.

After detailing the design and characterization of our plasma-liquid systems, we illustrate their applications to plant fertilization and wastewater disinfection. In a four-week collaborative experiment with the NCSU greenhouse, plants that received plasma-treated water grew significantly larger than plants that received tap water. This is directly attributable to the approximately hundred mg/L of NO$_3^-$ dissolved into solution by the plasma. The VHF source also proved effective at removing several aqueous contaminants designated harmful to humans by the EPA. Air plasma treatment of solutions contaminated with 1,4-dioxane showed log reduction times competitive with other advanced oxidative processes (AOP). Argon treatment of dixoane was an order of magnitude more effective in terms of log reduction time, although the associated costs are significantly higher. Perfluorooctanesulfonic acid (PFOS) proved resistant to several VHF design iterations. However, the water electrode design introduced in the passage above achieved a log reduction in low level PFOS concentrations over the course of twenty five minutes, suggesting that it may be viable as an advanced technology for degradation of persistent perfluorinated compounds.

Future work will serve to further unify the modelling and experimental work presented here. Three dimensional models are being actively developed to explore the spreading of discharges over water as a function of the water conductivity. Electromagnetic models are also being introduced into Zapdos that will enable simulation of the VHF source in contact with liquids. Completion of those models will be followed by simulations attempting to reproduce the gaseous OH density and liquid phase NO$_x^-$ and H$_2$O$_2$ concentration measurements.

\end{abstract}


%% ---------------------------- Copyright page ------------------------------ %%
%% Comment the next line if you don't want the copyright page included.
\makecopyrightpage

%% -------------------------------- Title page ------------------------------ %%
\maketitlepage

%% -------------------------------- Dedication ------------------------------ %%
\begin{dedication}
This is dedicated to my Mom, Dad, and Sister. Without them, I would never have gotten here. They've supported me through the good times and the bad. I couldn't have been blessed with a more loving family. This degree means a lot, but they will always mean immeasurably more.
\end{dedication}

%% -------------------------------- Biography ------------------------------- %%
\begin{biography}
Alexander David Lindsay was born in Seattle, WA on December 11, 1987 to Janet and Tom. Little sister Jessica arrived five and a half years later and has been a nuisance ever since. Alexander attended the University of Washington and obtained his bachelor's degree in chemical engineering in August of 2010. He began his doctoral studies at North Carolina State University in August 2011. After receiving his doctorate, Alexander will commence post-doctoral studies at the University of Illinois in the National Center for Supercomputing Applications.
\end{biography}

%% ----------------------------- Acknowledgements --------------------------- %%
\begin{acknowledgements}
I would like to thank my whole dissertation committee for teaching me so much about my field. Special thanks goes to Dr. Graves for hosting me in his lab for a full year and for his invaluable recommendations for modeling plasma physics. His theoretical knowledge in our field is unmatched. My greatest thanks goes to Dr. Shannon, who has not only been the most fantastic academic advisor I could have ever asked for but also a tremendous friend. Our relationship is hopefully just beginning.

Thanks to Jessie Lindsay for being as big a participant in the first four years of my journey here as anyone. She saw through things that a lot of other people wouldn't have been able to and made me a much better person in the process. Thanks also to Kyle Weinfurther, David Peterson, and Ben Daniel for helping lift me out of a tough time that overlapped with the most important stage of my research.

Thanks to all the members of the Shannon and Graves group that I had a chance to work and hang out with, particularly Brandon Byrns who mentored me for my first two years, teaching me so much about the VHF source and atmospheric plasmas. Life's pretty good when your lab partner is also one of your closest friends.

I would like to thank Brandon Curtis for introducing me to the open source way of thinking. It's revolutionized my research and career plans in the best possible way. I know now that I can learn just about anything. Thanks also goes to Jannis Teunissen for greatly extending my knowledge of open source tools and for teaching me how to think about modelling problems. I sincerely hope we can collaborate in the future. Finally, deep thanks goes out to the Moose development team and community who have taught me so much about computer science, math, physics, and engineering. It's been a great ride.
\end{acknowledgements}


\thesistableofcontents

\thesislistoftables

\thesislistoffigures
