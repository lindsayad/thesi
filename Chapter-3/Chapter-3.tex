\chapter{ZAPDOS: A TOOL FOR FULLY COUPLED MODELLING OF PLASMA-LIQUID SYSTEMS}
\label{chap:zapdos}

In \cref{chap:basic_science} we introduced two models addressing various aspects of plasma-liquids. In \cref{sec:plasfree_model}, we addressed momentum, heat, and neutral species transport using a 2D-axisymmetric model without explicity simulating the plasma discharge. In \cref{sec:plasliq} we examined the discharge physics coupled to the liquid phase in one dimension. The former model was created with the commercially available multi-physics package Comsol. However, efficient solution of the discharge governing equations coupled with liquid phase required creation of a novel plasma-simulation code, Zapdos, based on top of the Multiphysics Object-Oriented Simulation Environment (MOOSE). Creation of the fully-coupled plasma-liquid simulation environment is detailed in this chapter. In \cref{sec:zapdos} we describe Zapdos, the application physics code. In \cref{sec:moose} we detail changes implemented in the MOOSE framework allowing physics coupling across domains like that encountered in plasma-liquids.

\section{Zapdos Code}
\label{sec:zapdos}

\subsection{Zapdos Intro}
\label{sec:zapdos_intro}

It is perhaps prudent to warn the reader that the following subsection contains an overview of a lot of mathematics. We feel that this overview is worthwhile, however, for a couple of reasons. Firstly, the MOOSE application programmer principally responsible for providing two functions in his code: residuals and Jacobians. The following overview defines the meaning and importance of residuals and Jacobians in a physics simulation context. Secondly, the overview should give the reader an idea of the stunning level of control available to the MOOSE application programmer as he assembles and solves his physical simulation. We do not see it this way, but the high degree of control can be regarded as a two-edged sword. Building an accurate and efficient physical simulation on top of the MOOSE framework demands a level of understanding of the underlying mathematics well beyond that of a user of commercial multi-physics software. A minimum of a few months is required before acquiring the knowledge needed to achieve any interesting modelling results. However, once the developer has invested the requisite time and effort, the return is substantial as hopefully demonstrated in \cref{sec:plasliq} and in the exciting possibilities for future research (\cref{chap:conclusion}).

Zapdos is built on top of the MOOSE \cite{mooseSite} and libMesh \cite{libmeshSite} codes. MOOSE employs finite element methods (FEM), either Continuous Galerkin, Discontinuous Galerkin, or a combination, to solve fully coupled systems of partial differential equations (PDEs). Physics may also be segregated using the MOOSE multi-app system. After using FEM to discretize the governing equations, MOOSE interfaces with the code PetSc \cite{petscSite} to solve the non-linear or linear system of algebraic equations, $\vec{F}(\vec{u})$, via some form of Newton's method, where $\vec{F}$ is the residual vector and $\vec{u}$ is the vector of unknowns. A block diagram showing the interplay between Zapdos, MOOSE, libMesh, PetSc, as well as meshing and visualization packages (discussed more in \cref{sec:zap_meshing,sec:zap_post}) is shown in \cref{fig:diagram_workflow}. We will outline briefly the concept of forming the residual in the context of the finite element method. Let us consider the following differential equation:

\begin{equation}
  \nabla\cdot-D\nabla u = s(\vec{r})
  \label{eq:example}
\end{equation}

where diffusion of species $u$ is balanced by a source term $s(\vec{r})$. We refer to \cref{eq:example} as the strong form of our reaction-diffusion example problem. For FEM we convert the problem to what is known as the weak form by multiplying by a test function $\psi_i$ and integrating over the whole domain. We also move all terms to the left-hand side (LHS) of the equation such that the right-hand side (RHS) becomes 0. Our problem now becomes:

\begin{equation}
  \int_{\Omega}\psi_i\nabla\cdot-D\nabla u\,dV - \int_{\Omega}\psi_is(x)\,dV = 0
  \label{eq:example_weak}
\end{equation}

where $\Omega$ represents the extent of the domain. The first term is then integrated by parts to gives us:

\begin{equation}
  \int_{\Omega}\nabla\psi_i\cdot D\nabla u\,dV - \int_{\partial\Omega}\psi_iD\nabla u \cdot \vec{n}\,dS -\int_{\Omega}\psi_is(x)\,dV = 0
  \label{eq:example_weak_parts}
\end{equation}

where $\partial\Omega$ represents the domain boundaries. Integrals over the domain, e.g. terms 1 and 3 in \cref{eq:example_weak_parts}, are written in Zapdos and MOOSE as kernel class objects. Integrals over domain boundaries are written as boundary condition class objects. We will discuss Zapdos class objects more shortly. To continue our development of FEM, we must discretize the problem, or in other words convert our integral-differential equation into an algebraic equation. We accomplish this by dividing the domain volume into individual elements and associating basis functions with different geometric aspects of the discretization. Division of the domain into elements is also referred to as meshing. An example of a two-dimensional mesh is shown in \cref{fig:mesh}, where the domain has been sub-divided into triangular elements. For a two-dimensional domain, basis functions can be associated with mesh elements, edges, or vertices; in three-dimensions: elements, faces, edges, and vertices. Most commonly, Zapdos uses a linear basis associated with mesh vertices; these basis functions are commonly referred to as linear Lagrange. We construct a basis function $\phi_j(x,y,z)$ for each vertex that exists on the mesh (j = 1, 2, ..., N where N is the number of vertices in the domain). The function $\phi_j$ has these important properties: it is equal to one on vertex $j$ with coordinates $(x_j,y_j,z_j)$ and zero on all other vertices. This is illustrated in \cref{fig:basis_function} for a vertex sitting at the intersection of four two-dimensional triangular elements.

\begin{figure}[htbp]
  \centering
  \includegraphics[width=\maxwidth{\textwidth}]{test.png}
  \caption{Block-diagram laying out simulation work-flow}
  \label{fig:diagram_workflow}
\end{figure}

\begin{figure}[htbp]
  \centering
  \includegraphics{2d_mesh.png}
  \caption{An example two-dimensional mesh. The domain is sub-divided into triangular elements.}
  \label{fig:mesh}
\end{figure}

\begin{figure}[htbp]
  \centering
  \includegraphics{basis_function.png}
  \caption{Basis function $\phi_1$ associated with vertex one. Vertex one sits at the intersection of four triangular two-dimensional elements. Note that $\phi_1$ equals unity at $(x_1,y_1)$ and zero at all other vertices. Illustration taken from \cite{FlahertyFEA}}
  \label{fig:basis_function}
\end{figure}

The dependent variable $u$ is obtained by summing over the whole basis:

\begin{equation}
  u = \sum_{j=1}^Nu_j\phi_j(x,y,z)
  \label{eq:basis_sum}
\end{equation}

where $u_j$ is the value of u at vertex $j$. The set of $u_j's$ represent the degrees of freedom or the quantities to be solved for. It is worthwhile to note that in Galerkin FEM, which is the only method used in Zapdos, the following relationship exists between basis and test functions: $\phi_k=\psi_k$ for k = 1, 2, ..., N. Substitution of \cref{eq:basis_sum} into \cref{eq:example_weak_parts} yields

\begin{equation}
  \int_{\Omega}\nabla\psi_i\cdot D\nabla \sum_{j=1}^Nu_j\phi_j(x,y,z)\,dV - \int_{\partial\Omega}\psi_iD\nabla \sum_{j=1}^Nu_j\phi_j(x,y,z) \cdot \vec{n}\,dS -\int_{\Omega}\psi_is(x)\,dV = R_i \approx 0
  \label{eq:example_weak_final}
\end{equation}

where we have introduced the idea that our governing equations form residual statements ($R_i$) that we hope to eventually drive to zero. The first and second terms of \cref{eq:example_weak_final} arising from diffusion can be analytically integrated. To complete the transformation of \cref{eq:example_weak_final} into an algebraic equation, the source term integral is computed with numerical quadrature. MOOSE employs Guassian quadrature, which for a general integrand s(x), can be written as:

\begin{equation}
  \int_a^bs(x)\,dx = \frac{b-a}{2}\sum_{i=1}^nw_is\left(\frac{b-a}{2}x_i+\frac{a+b}{2}\right)
  \label{eq:quad}
\end{equation}

where the $w_i's$ and $x_i's$ are the quadrature weights and points respectively. Details of calculation of $w_i$ and $x_i$ are not important to our purpose here; it is enough to know the concept of how a general integral is converted into an algebraic expression. After numericall integrating \cref{eq:example_weak_final}, the resulting algebraic expression can be evaluated using an initial guess for the $u_j's$. This results in the initial residual, with the residual essentially representing how close the $u_j's$ are to satisfying the governing equations; a residual of zero would represent a numerically perfect solution. In practice, the initial guess may be far from the actual solution. To rectify this, Zapdos, through MOOSE's interface with PetSc, uses Newton's method to iterate the solution vector $\vec{u}$ towards the true solution. Newton's method derives from a Taylor expansion: \cite{knoll2004jacobian}:

\begin{equation}
  \vec{R}(\vec{u}_{k+1}) = \vec{R}(\vec{u}_k) + \tilde{R'}(\vec{u}_k)(\vec{u}_{k+1}-\vec{u}_k)+\text{higher-order terms}
  \label{eq:Newton_step1_deriv}
\end{equation}

where in our case $\vec{R}$ is the residual vector of length N and $\vec{u}$ is the solution vector, also of length N where N in our example problem is equal to the number of mesh vertices as well as the number of basis functions. The index $k$ denotes the $k^{th}$ iteration of the Newton solve. In a more general case we may be solving for multiple fields, e.g. in addition to a concentration variable $u$ we may be solving for a temperature variable $T$. In this more general case (still limiting ourselves to a linear Lagrange discretization), $\vec{R}$ and $\vec{u}$ will have dimension equal to the \# of mesh vertices times the \# of field variables with $\vec{u}$ comprised of $T_j's$ and $u_j's$. (Indices $j$ and $k$ should not be confused; $u_j$ is one scalar element of the solution vector $\vec{u}$; $\vec{u}_k$ is the $k^{th}$ iterate of the solution vector computed during the Newton solve.) Returning to \cref{eq:Newton_step1_deriv}, if we set the left-hand side to zero (our goal is always to drive $\vec{R}_{k+1}$ to zero) and neglect the higher-order terms we reproduce the strict Newton method:

\begin{equation}
  \tilde{J}(\vec{u}_k)\vec{\delta u}_k = -\vec{R}(\vec{u}_k)
  \label{eq:Newton}
\end{equation}
\begin{equation}
  \vec{u}_{k+1} = \vec{u}_k + \vec{\delta u}_k
  \label{eq:strict_Newton}
\end{equation}

where $\tilde{J}$ is the Jacobian matrix formed by taking the derivatives of the residual vector with respect to the solution vector (equivalent to $\tilde{R'}$ in \cref{eq:Newton_step1_deriv}).\cite{knoll2004jacobian}  We iterate with \cref{eq:Newton,eq:strict_Newton} until the norm of $\vec{R}$ is below some user-defined tolerance. Strict Newton is known to have locally quadratic convergence; e.g. if the initial guess is relatively close to the solution, Newton's method will converge quickly and with ease. \cite{dennis1996numerical} In practice, however, the initial guess may not be close the solution. In this case a globalization method is used to increase the rate of convergence. Line search and trust region methods are the most common techniques for globalization. The default globalization in MOOSE applications is a line search with cubic backtracking. In a line search, \cref{eq:strict_Newton} is modified by simply inserting a multiplicative factor $\lambda$ in front of $\vec{\delta u}_k$ as shown in \cref{eq:line_search}.

\begin{equation}
  \vec{u}_{k+1} = \vec{u}_k + \lambda\vec{\delta u}_k
  \label{eq:line_search}
\end{equation}

Determining $\lambda$ is the realm of the particular line search technique chosen. The cubic backtracking technique for chosing $\lambda$ that is used in all Zapdos simulations is described in detail in \cite{dennis1996numerical}. We will give a short summary here. For simplicity, let us consider minimizing the residual $R$ of one nonlinear equation with one degree of freedom, $u$. We use the index $k$ to denote the $k^{th}$ Newton iteration and we introduce the index $l$ to denote the $l^{th}$ attempt to select a good value for $\lambda$. Since Newton's method displays exceptional convergence when $u_k$ is sufficiently close to the true solution, the zeroth guess for $\lambda$ is always $\lambda_0 = 1$. If $\lambda_0$ does not yield a desirable result (see \cref{eq:stopping}, then $\lambda_1$ is determined through quadratic backtracking (for details see \cite{dennis1996numerical}). If $\lambda_1$ also fails, the line search algorithm has sufficient information to perform a cubic backtrack. Cubic backtracking has the benefit over quadratic backtracking of being able to minimize $R$ when $R$ has negative curvature in the vicinity of $u_k$. Cubic backtracking minimizes the equation:

\begin{equation}
  m_{cu}(\lambda_l) = a\lambda_l^3+b\lambda_l^2+J(u_k)\delta u_k\lambda_l+R(u_k)
  \label{eq:bt}
\end{equation}

where J is our one-dimensional Jacobian equal to $\frac{\partial R}{\partial u}$ and $\delta u_k$ is the full Newton step determined from \cref{eq:Newton}; $a$ and $b$ are given by: \cite{dennis1996numerical}

\begin{equation}
  \begin{bmatrix} a \\ b \end{bmatrix} = \frac{1}{\lambda_{l-1}-\lambda_{l-2}} \begin{bmatrix} \frac{1}{\lambda_{l-1}^2} & \frac{-1}{\lambda_{l-2}^2} \\ \frac{-\lambda_{l-2}}{\lambda_{l-1}^2} & \frac{\lambda_{l-1}}{\lambda_{l-2}^2} \end{bmatrix} \begin{bmatrix} R(u_k+\lambda_{l-1}\delta u_k) - R(u_k) - J(u_k)\delta u_k\lambda_{l-1} \\ R(u_k+\lambda_{l-2}\delta u_k) - R(u_k) - J(u_k)\delta u_k\lambda_{l-2} \end{bmatrix}
  \label{eq:dets_bt}
\end{equation}

The stopping criterion used for $\lambda$ is: \cite{dennis1996numerical}

\begin{equation}
  R(u_k+\lambda_l\delta u_k) \leq R(u_k) + \alpha\lambda_lJ(u_k)\delta u_k
  \label{eq:stopping}
\end{equation}

with $\alpha$ chosen to be some value between 0 and 1; the default in PetSc is $\alpha=10^{-4}$. When the criterion of \cref{eq:stopping} is met, then the solution $u$ is updated through \cref{eq:line_search}. \Cref{eq:stopping} essentially requires that the average rate of decrease in R when moving from from $u_k$ to $u_k+\lambda_l\delta u_k$ be some fraction of the initial rate of decrease in that direction. The combination of \cref{eq:Newton,eq:line_search} is properly called a quasi-Newton method since the full Newton step is not necessarily used to update the solution vector at each iteration. However, since a globalization method like the line search described above is always combined with \cref{eq:Newton} in Zapdos simulations and any other practical Newton implentation, we will drop the quasi- prefix for the remainder of this discussion.

The linear problem for determining $\delta u_k$, \Cref{eq:Newton}, may be solved through either direct or iterative methods. The details of these methods are not critical to understanding the work in this dissertation, so we will give only a brief summary of them here. A direct method like lower-upper decomposition works well for relatively small problems. However, larger problems, particularly problems in three dimensions, require iterative Krylov subspace methods to be feasible. \cite{heil2008solvers} The most common Krylov technique and the one used by default in Zapdos is the generalized minimum residual method (GMRES) with restart. GMRES is an Arnoldi-based method; its algorithm is given in \cite{saad1986gmres}. The option to restart the GMRES algorithm prevents the inherent problem that as the number of linear iterations $k$ increases, the number of multiplications required to solve the linear system increases by $\frac{1}{2}k^2N$ where N denotes the number of degrees of freedom being solved for. \cite{saad1986gmres} The GMRES algorithm only requires knowledge of matrix vector products to complete the iteration, e.g. the Jacobian solution vector product on the LHS of \cref{eq:Newton}, as opposed to individual elements of the matrix. \cite{knoll2004jacobian} This makes GMRES conducive for a class of methods known as Jacobian-free Newton-Krylov (JFNK) methods, which will be discussed more shortly. Convergence of iterative methods like GMRES tends to improve as the condition number of the matrix $\tilde{A}$ in the equation $\tilde{A}\vec{x} = \vec{b}$ decreases. The condition number can be decreased through a process called preconditioning. Right preconditioning solves the new problem: $\tilde{A}\tilde{P}^{-1}\tilde{P}\vec{x} = \vec{b}$; left preconditioning solves problem $\tilde{P}^{-1}(\tilde{A}\vec{x} - \vec{b}) = 0$. The preconditioning matrix $\tilde{P}$ is usually based on the matrix $\tilde{A}$ or in our case the Jacobian matrix $\tilde{J}=\tilde{A}$. In creating $\tilde{P}$, there is a balance to strike between the degree of conditioning and the computational cost of applying the preconditioner. Choosing $\tilde{P}=\tilde{J}$ results in a conditioned matrix $\tilde{P}^{-1}\tilde{J}=\tilde{I}$ with an optimal condition number of one; solving the resulting conditioned system will take only one iteration to converge. However, the cost of applying the preconditioner is at a maximum. In practice, $\tilde{P}$ is much more sparse than $\tilde{J}$ and can be formed through a variety of techniques. The default serial preconditioning method for iterative linear solutions with Zapdos is incomplete lower-upper factorization which is described in detail in \cite{dupont1968approximate} and \cite{saad1996iterative}. Other preconditioning techniques are more parallelizable including block-Jacobi, which only fills the diagonal elements of the preconditioning matrix; \cite{saad1996iterative} the domain decomposition technique, additive Schwarz, which operates under the principle of divide and conquer; \cite{saad1996iterative} and multi-grid methods that rely on mesh restriction and prolongation operators. \cite{knoll2004jacobian} We recommend consulting this passage's references for more details on the mentioned preconditioning techniques.

When analytic calculation of Jacobian elements becomes either impossible or impractical, the property that GMRES only requires knowledge of matrix-vector products as opposed to individual matrix elements becomes very useful. It allows the modeller to implement the Jacobian-free approximation: \cite{knoll2004jacobian}

\begin{equation}
  \tilde{J}\vec{v} \approx \frac{\vec{R}(\vec{u}+\epsilon\vec{v})-vec{R}(\vec{u})}{\epsilon}
  \label{eq:jacob_free}
\end{equation}

where $\epsilon$ is a finite-differencing parameter chosen by the modeller. Using this approximation, the elements of the Jacobian matrix are never explicitly formed, giving the algorithm its ``Jacobian-free'' moniker. Using JFNK, one achieves Newton-like nonlinear convergence without the need for computing or storing the true Jacobian. \cite{knoll2004jacobian} Through finite differencing of the residual, JFNK feels out the true Jacobian, generally giving better convergence than if a Newton-Krylov method is used in conjunction with an incomplete or incorrect user-provided Jacobian matrix. Although JFNK methods are sometimes referred to as ``matrix-free'' methods, this is somewhat of a misnomer. In practice JFNK almost always involves forming a preconditioning matrix since the linear solve is intrinsically iterative.

In the precending section, we have hopefully illustrated the generation and importance of residual and Jacobian statements in simulating our physical system. To summarize, residual statements are pieces of the physical governing equations cast in the weak form and discretized with the finite element method. The Jacobian matrix $\tilde{J}$ is composed of derivatives of the residuals with respect to the solution vector $\vec{u}$. A maximally efficient application code in terms of computational time will have a complete and correct set of Jacobian statements and will employ a Newton method globalized with a line search. Small problems with accurate Jacobians can be solved directly, via lower-upper factorization for example; however, higher-dimensional problems may require a preconditioned iterative method to solve the linear system in \cref{eq:Newton}. If developer time is at a premium or some residual derivatives cannot be computed analytically, some Jacobian statements can be omitted and the Jacobian-Free approximation of \cref{eq:jacob_free} can be used. While still efficient, this method displays slower convergence than if a fully analytic and accurate Jacobian is supplied. Thus, the low-temperature plasma application Zapdos supplies complete and correct Jacobian statements wherever ot can. As new physics are introduced into Zapdos, newly coded analytical Jacobians are compared against PetSc Jacobians formed through finite differencing of the residual statements to ensure accuracy. The one-dimensional simulations run in \cref{sec:plasliq} featured fully accurate Jacobians; as a result, Newton's method combined with a direct solve of the linear system displayed the best rate of convergence. New 2-D axisymmetric simulations not reported in this work feature boundary condition residuals with non-local integrals. The coding infrastructure required to implement the corresponding Jacobian functions does not currently exist in MOOSE. Since the Jacobian is not completely accurate in this case, direct Newton and preconditioned JFNK display comparable rates of convergence.

Having given an overview of the mathematics that Zapdos relies on, we will turn to its coding implementation. Zapdos partitions governing equation terms into individual pieces called kernels. Each kernel contains the residual and the corresponding Jacobian statements. Recall our diffusion-reaction example from \cref{eq:example,eq:example_weak,eq:example_weak_parts,eq:example_weak_final}. Let us consider the first term in \cref{eq:example_weak_parts}; the corresponding Zapdos code looks like:

\begin{lstlisting}[language=C++,caption = Example of residual and Jacobian function definitions, label = code:resid_jacob]
Real
CoeffDiffusionLin::computeQpResidual()
{
  // Computes the residual
  return -_diffusivity[_qp] * _grad_u[_qp] * _r_units * -_grad_test[_i][_qp] * _r_units;
}

Real
CoeffDiffusionLin::computeQpJacobian()
{
  // Computes the Jacobian
  return -_diffusivity[_qp] * _grad_phi[_j][_qp] * _r_units * -_grad_test[_i][_qp] * _r_units;
}
\end{lstlisting}

where \_diffusivity is the diffusion coefficient, \_u is our solution variable, \_phi and \_test represent the shape and test functions that we introduced above, and \_qp represent the positions of quadrature points used for numerical integration. By splitting governing equations in this way into individual terms/kernels, code reproduction is kept at a minimum; analagous terms can be used in many different settings, e.g. a ``diffusion'' term has the exact same mathematical form as a ``conduction'' or ``viscosity'' term and so the same kernel code can be used for all three physics cases. Material properties like mobilty and diffusivity are defined in a materials file separated from the kernel code. Material properties can be defined as constants, as functions of the solution variables, or as properties to be read from look-up tables. Through MOOSE, Zapdos provides an interface for linear, bilinear, and spline interpolation of material properties. Boundary conditions are available in ``Nodal'' and ``Integrated'' flavors. Nodal boundary conditions are dirichlet like conditions that are enforced strongly. Integrated boundary conditions are cast in the weak form and often arise from performing integration by parts on divergence terms in the governing equations.

As of commit f74bad6, Zapdos has the necessary kernels and boundary conditions for solving gas phase DC discharge fluid models as well as conventional convection-diffusion-reaction equations for dilute species in a fluid. (Zapdos uses the software ``git'' for version control. The commit ``hash'' f74bad6 is essentially a version number.) Another student is working on implementing RF plasma simulation capabilities (for capacitively coupled plasmas this will only require slight modification of some boundary conditions; inductively coupled plasmas will require a little more work). These efforts will be detailed further in \cref{chap:conclusion}.

Zapdos solutions are output to an exodus file by default, although MOOSE provides varying levels of support for some other output file formats (including full support for simple CSV). Exodus files are a file type developed at Sandia National Lab designed specifically for storing and retrieving finite element data. \cite{schoof1994exodus} These exodus files are most commonly viewed graphically with either of the open source packages Visit or Paraview. For users more programatically inclined, Paraview provides python tools that enable the user to directly read the exodus file and create publication level plots in MatPlotLib with a single script (as is done for a lot of figures in this dissertation). For transient simulations, results for any solution or auxiliary variable can be viewed while the calculation is on-line. Results are also not lost if a solve is cancelled for any reason. These features enable quick convergence debugging of a failing or failed solve.

Another feature of Zapdos is the adaptive mesh refinement inherited from MOOSE. The user can choose from several different indicators, including the jump in a solution gradient or laplacian between elements, for determing where mesh refinement should take place. Figures \ref{fig:step15} and \ref{fig:step49} show the results of an advection-diffusion simulation for temperature in which a pressure difference between the left and right ends of the domain induce bulk flow from left to right. The effective mobility of the temperature is specified to be greater in the bottom half of the domain, resulting in faster temperature flow in the bottom half. Note that the mesh is refined at the head of the temperature flow where numerical instabilities are more likely to occur; once the temperature front has passed the mesh is coarsened to reduce computational expense. This feature can be incredibly useful when trying to track ionization bullets (\cref{fig:bullets}) or similar phenomena.

\begin{figure}[htbp]
  \centering
  \includegraphics[width=.4\textwidth]{Time_step_15.eps}
  \caption{Propagating front. Time step 15. Note how the mesh is fine around the solution gradients and coarse elsewhere.}
  \label{fig:step15}
\end{figure}

\begin{figure}[htbp]
  \centering
  \includegraphics[width=.4\textwidth]{Time_step_49.eps}
  \caption{Propagating front. Time step 49. Note how the mesh is fine around the solution gradients and coarse elsewhere.}
  \label{fig:step49}
\end{figure}

\begin{figure}[htbp]
  \centering
  \includegraphics[width=\textwidth]{Ionization_bullets.png}
  \caption{Ionization bullets simulated with Zapdos. Mesh adaptivity is used to follow their propagation.}
  \label{fig:bullets}
\end{figure}

\subsection{Zapdos Kernels}
\label{sec:zap_kernels}

As mentioned in the introductory section, Zapdos takes each term in a governing equation and casts that term as a class with methods for computing the residual and Jacobian. These governing equation term classes are called kernels. As of commit f74bad6, Zapdos has 77 kernels. However, not all of these have utility, e.g. one kernel may have been re-cast as a new class without the old class being removed from the kernel directory. The most important kernels, e.g. the ones actively being used for physics and engineering research, are enumerated below. An important feature of Zapdos is the option to cast concentration or density variables in a logarithmic form, e.g. $N_k = \ln\left(n_k\right)$ where $N_k$ is the logarithmic variable representation of the density and $n_k$ is the true physical density. This is done for the modelling studies presented in \cref{sec:plasliq}. The advantage of the logarithmic casting is that it prevents the true concentration from ever becoming negative. Negative concentrations can be a product of and contribute to numerical instabilities. Negative concentrations can cause source terms to become sink terms and visa versa, thus it is advantageous to avoid them if possible.

Also for all the simulations described in \cref{sec:plasliq}, it is the logarithm of the \textit{product} of the electron density and mean energy that is a solution variable as opposed to simply the logarithm of the mean energy. Thus for the simplest plasma discharge simulation, there are four solution variables:

\begin{gather}\label{eq:soln_vars}
  N_i = \ln n_i\\
  N_e = \ln n_e\\
  E_n = \ln\left(n_e\epsilon\right)\\
  V = V
\end{gather}

where n$_i$ and n$_e$ are the physical ion and electron densities respectively, $\epsilon$ is the mean electron energy, and V is the potential. Anywhere that $n_i$ exists in the governing equations, it is replaced with $e^{N_i}$; $n_e$ is replaced with $e^{N_e}$; the product of $n_e$ and $\epsilon$ is replaced with $e^{E_n}$. Whatever units are used for the original variables are retained by their replacement expressions, e.g. if $n_e$ has units of \#/m$^3$ then the expression $e^{N_e}$ has units of \#/m$^3$. While the choice to use the product of n$_e$ and $\epsilon$ simplifies some parts of the governing equation, it complicates others. In particular the electron transport and Townsend coefficients that are functions of the mean electron energy become functions of two solution variables, N$_e$ and E$_n$, as opposed to just one. Thus residual/governing equation terms that involve electron transport or Townsend coefficients must include Jacobian contributions from both N$_e$ and E$_n$.

In order to improve conditioning of the Jacobian, Zapdos provides the user options for unit scaling. The potential can either be cast in units of volts or kilovolts (this will be made even more flexible in the future). This choice is made in the GlobalParams block of the Zapdos input file, by specifying $potential\_units = V$ or $potential\_units = kV$. The length units can also be scaled. In the coupled plasma-liquid simulations described in \cref{sec:plasliq}, the plasma domain length is 1 mm, whereas the water domain length is 100 nm. Thus at the beginning of the input file, in order to scale closer to unity, we specify $dom0scale = 1e-3$ and $dom1scale = 1e-7$. The values of $dom0Scale$ and $dom1Scale$ can then be accessed using the syntax, \$\{\}. \cite{GetPot} Thus for any kernel or boundary condition that contains a gradient term, we specify the parameter: $position\_units = \$\{dom0Scale\}$ or $position\_units = \$\{dom1Scale\}$ depending on whether the kernel or boundary condition is acting in the gas or liquid phase. Unlike the $potential\_units$ parameter which must be a string equal to $V$ or $kV$, the $position\_units$ parameter can be set to any real number. The length scaling is represented in \cref{tab:zap_kernels} as the symbol $l_c$ where $l_c$ is actually equal to $1/position\_units$; the potential scaling is represented by $V_c$ where $V_c=1000$ if $potential\_units=kV$ else $V_c=1$ if $potential\_units=V$.

Zapdos tries to be as generic and modular as possible in the formulation of its kernels, e.g. a diffusion kernel code should be as applicable to the diffusion of Argon ions in the gas phase as it is to neutral OH radicals in the liquid phase. However, advection-diffusion-reaction (ADR) terms for electrons in the plasma typically have to have their own kernels because the transport and Townsend coefficients are allowed to be functions of the mean electron energy as opposed to constants. In this case Jacobian terms must be provided for the coefficient functional dependence on both N$_e$ and E$_N$. This functional dependence does not exist when the coefficients are constant as in the cases where we are modelling transport of heavy ions and neutrals, thus those kernels can be completely generic. For generic kernels, the solution variable is denoted by $u$ in \cref{tab:zap_kernels}; a coupled solution variable is denoted by $v$. Note that the kernel residuals/governing equation terms are cast in the weak form, i.e. each term in the governing equation is multiplied by a test function $\psi_i$ where $i$ denotes the $i^{th}$ shape function. Divergence terms, e.g. flux terms, are integrated by parts to produce both a volumetric kernel term and a surface boundary condition term (or surface discontinuous Galerkin term if a discontinuous Galerkin discretization is used). $\mu$ is the mobility of the species the kernel is acting on, $D_k$ is the diffusivity, $l_c$ is the characteristic length defined in the paragraph above, $\alpha_{iz}$, $\alpha_{ex}$, and $\alpha_{el}$ are the Townsend ionization, excitation, and elastic collision coefficients respectively, $sgn(q)$ is the charge sign, $k$ is used to generally represent reaction coefficients, $N_A$ is Avogadro's number, and $e$ is the Coulombic charge equal to $1.6x10^{-19}$ C. All other symbols should be defined in their corresponding table entry.

\begin{ThreePartTable}
  \begin{TableNotes}
    %% \footnotesize
    \item [a] \(\displaystyle \vec{\Gamma_e} = \mu_e(N_e,E_n) \nabla V l_c e^{N_e} - D_e(N_e,E_n) e^{N_e} \nabla N_e l_c\)
  \end{TableNotes}

  \begin{longtable}{>{\centering}m{2.25in}| >{\centering}m{1.75in}| >{\raggedright\arraybackslash}m{2in}}
    %% \hline\hline
    %% This goes at & the top of the & first heading\\
    %% \hline\hline
    %% \endfirsthead
    \textbf{Kernel Name} & \textbf{Governing Eqn. Term}\tnote{a} & \textbf{Description}\\\hline\hline
    \endhead
    \caption{Kernels in Zapdos used for simulations presented in \cref{sec:plasliq}}
    \endfoot
    \caption{Kernels in Zapdos used for simulations presented in \cref{sec:plasliq}}
    \label{tab:zap_kernels}
    \endlastfoot

    ElectronTimeDerivative & \(\displaystyle \psi_i e^u\frac{\partial u}{\partial t}\) & Generic accumulation term\\\hline
    EFieldAdvectionElectrons & \(\displaystyle -\nabla\psi_i \mu_e(N_e,E_n) e^{N_e}\nabla V /l_c^2\) & Electron specific electric field driven advection term\\\hline
    CoeffDiffusionElectons & \(\displaystyle \nabla\psi_i D_e(N_e,E_n) e^{N_e} \nabla N_e /l_c^2\) & Electron specific diffusion term\\\hline
    ElectronsFromIonization & \(\displaystyle -\psi_i \alpha_{iz}(E_n,N_e) \lvert\vec{\Gamma_e}\rvert\) & Rate of production of electrons from ionization\\\hline
    LogStabilizationMoles & \(\displaystyle -\psi_i e^{-(b+u)}\) & Kernel stabilizes solution variable $u$ in places where $u\rightarrow 0$; $b$ is the offset value specified by the user. A typical value for $b$ is 20.\\\hline
    EFieldAdvection & \(\displaystyle -\nabla \psi_i \mu\sgn(q) e^u \cdot -\nabla V /l_c^2\) & Generic electric field driven advection term\\\hline
    CoeffDiffusion & \(\displaystyle -\nabla \psi_i -D_e e^u \nabla u /l_c^2\) & Generic diffusion term\\\hline
    ReactantFirstOrderRxn & \(\displaystyle \psi_i k e^u\) & Generic first order reaction sink term for $u$ ($u$ is the reactant); $k$ is the reaction rate coefficient\\\hline
    ReactantAARxn & \(\displaystyle 2 \psi_i k e^{2u}\) & Generic second order reaction sink term for $u$ in which two molecules of $u$ are consumed\\\hline
    CoeffDiffusionLin & \(\displaystyle -\nabla \psi_i \cdot -D \nabla u /l_c^2\) & Generic \textit{linear} diffusion term, e.g. this is a diffusion term for solution variables \textit{not} cast in a logarithmic form\\\hline
    ChargeSourceMoles\_KV & \(\displaystyle \frac{-\psi_i e \sgn(q) N_A e^v}{V_c}\) & Used for adding charged sources to Poisson's equation; $e^v$ represents the charged particle density of species $v$. This kernel assumes that densities are measured in units of mol/volume as opposed to \#/volume.\\\hline
    IonsFromIonization & \(\displaystyle -\psi_i \alpha_{iz}(E_n,N_e) \lvert\vec{\Gamma_e}\rvert\) & Same governing term/residual as ElectronsFromIonization; however, the Jacobian structure is different. $\frac{\partial R_i}{\partial N_e}$ will be on-diagonal for ElectronsFromIonization and off-diagonal for IonsFromIonization\\\hline
    ProductFirstOrderRxn & \(\displaystyle -\psi_i k e^v\) & Generic first order reaction source term for $u$ ($v$ is the reactant)\\\hline
    ProductAABBRxn & \(\displaystyle -2\psi_i k e^{2v}\) & Generic second order reaction source term in which two molecules of $v$ are produced from two molecules of $u$\\\hline
    EFieldAdvectionEnergy & \(\displaystyle -\nabla \psi_i \mu_{\epsilon}(N_e,E_n) e^{E_n} \cdot \nabla V /l_c^2\) & Electron energy specific electric field driven advection term\\\hline
    CoeffDiffusionEnergy & \(\displaystyle -\nabla \psi_i \cdot -D_{\epsilon}(N_e,E_n) e^{E_n}\nabla E_n /l_c^2\) & Electron energy specific diffusion term\\\hline
    JouleHeating & \(\displaystyle -\psi_i \nabla V V_c /l_c \cdot \vec{\Gamma_e}\) & Joule heating term for electrons\\\hline
    ElectronEnergyLossFromIonization & \(\displaystyle \psi_i \alpha_{iz}(N_e,E_n)\lvert\vec{\Gamma_e}\rvert E_{iz}\) & Electron energy loss term for inelastic ionization collisions; $E_{iz}$ is the energy lost in Volts in a single ionization collision\\\hline
    ElectronEnergyLossFromExcitation & \(\displaystyle \psi_i \alpha_{ex}(N_e,E_n)\lvert\vec{\Gamma_e}\rvert E_{ex}\) & Electron energy loss term for inelastic excitation collisions; $E_{ex}$ is the energy lost in Volts in a single excitation collision\\\hline
    ElectronEnergyLossFromElastic & \(\displaystyle \psi_i \alpha_{el}(N_e,E_n)\lvert\vec{\Gamma_e}\rvert \frac{3m_eT_e}{m_n}\) & Electron energy loss term for elastic collisions. $\alpha_{el}$ is the elastic Townsend coefficient; $m_e$ is the electron mass; $m_n$ is the mass of the neutral background gas; $T_e = \frac{2\epsilon}{3}$ is the electron temperature\\\hline
    \insertTableNotes
  \end{longtable}
\end{ThreePartTable}

\subsection{Zapdos Auxiliary Kernels}
\label{sec:zap_aux}

Zapdos implements a variety of auxiliary kernels that, while not essential to the solve, are important for visualizing and understanding the plasma-liquid physics. The auxiliary kernels that are employed for the simulations in \cref{sec:plasliq} are outlined in \cref{tab:aux_kernels}.

\begin{ThreePartTable}
  \begin{TableNotes}
    %% \footnotesize
    %% \item [a]
  \end{TableNotes}

  \begin{longtable}{>{\centering}m{2in}| >{\centering}m{2in}| >{\raggedright\arraybackslash}m{2in}}
    \textbf{AuxKernel Name} & \textbf{Expression} & \textbf{Description}\\\hline\hline
    \endhead
    \caption{AuxKernels in Zapdos used for visualization of simulation results described in \cref{sec:plasliq}}
    \endfoot
    \caption{AuxKernels in Zapdos used for visualization of simulation results described in \cref{sec:plasliq}}
    \label{tab:aux_kernels}
    \endlastfoot

    PowerDep & \(\displaystyle \sgn(q)e N_A\cdot(\sgn(q)\mu\cdot-\nabla V e^{N_k} - D e^{N_k} \nabla N_k) \cdot -\nabla V V_c/l_c^2\) & Amount of power deposited into a user specified specie by Joule Heating\\\hline
    ProcRate & \(\displaystyle N_A \cdot \abs{-\mu_e\cdot-\nabla V e^{N_e} - D_e e^{N_e} \nabla N_e} \cdot \alpha /l_c\) & Reaction rate for electron impact collisions in units of $\frac{\#}{m^3s}$. User can pass choice of elastic, excitation, or ionization\\\hline
    ElectronTemperature & \(\displaystyle \frac{2}{3}e^{E_n-N_e}\) & The electron temperature\\\hline
    Position & \(\displaystyle x l_c\) & Produces an elemental auxiliary variable useful for plotting against other elemental auxiliary variables. Mesh points automatically output by Zapdos only work for plotting nodal variables. Since almost all auxiliary variables are elemental, this AuxKernel is very important.\\\hline
    Density & \(\displaystyle e^{N_k} N_A\) & Returns physical densities in units of $\frac{\#}{m^3}$\\\hline
    Efield & \(\displaystyle -\nabla V / l_c\) & Returns the x-component of the electric field (only relevant component for 1-D simulations)\\\hline
    Current & \(\displaystyle \sgn(q_k) e N_A \cdot(\sgn(q)\mu_k\cdot-\nabla V e^{N_k} / l_c - D_k e^{N_k} \nabla N_k / l_c)\) & Returns the electric current associated with the flux of species $k$\\\hline
    EFieldAdvAux & \(\displaystyle \sgn(q_k)\mu_k e^{N_k} \cdot -\nabla V N_A / l_c\) & Returns the electric field driven advective flux of species $k$\\\hline
    DiffusiveFlux & \(\displaystyle -D_k e^{N_k} \nabla N_k N_A / l_c\) & Returns the diffusive flux of species $k$\\\hline
  \end{longtable}
\end{ThreePartTable}

\subsection{Zapdos Interface Kernels}
\label{sec:zap_interface}

Critical to fully coupling the plasma and liquid phase simulations are the conditions at the interface. Initially, it was not possible to create interfacial conditions in Zapdos because the capability did not exist in the MOOSE framework. However, as described in \cref{sec:moose}, we were able to add the capability to the framework, and thus it is now possible to create interfacial conditions in Zapdos. A couple important ones that are used in mean\_en.i (see \cref{sec:zap_input}) are described in \cref{tab:interface}.

\begin{ThreePartTable}

  \begin{TableNotes}
  \end{TableNotes}

  \begin{longtable}{>{\centering}m{2in}| >{\centering}m{2in}| >{\raggedright\arraybackslash}m{2in}}
    \textbf{InterfaceKernel Name} & \textbf{Expression} & \textbf{Description}\\\hline\hline
    \endhead
    \caption{Important InterfaceKernels in Zapdos}
    \endfoot
    \caption{Important InterfaceKernels in Zapdos}
    \label{tab:interface}
    \endlastfoot

    InterfaceAdvection & \(\displaystyle -\psi_{i,el}\mu_{k,n}\sgn(q_k) e^{N_{k,n}}\nabla V_n \cdot \vec{n} / (l_{c,n}l_{c,el})\) & Used to include the electric field driven advective flux of species $k$ into or out of a neighboring subdomain. The subscript $el$ denotes the subdomain to which the InterfaceAdvection residual is being added. The subscript $n$ denotes the neighboring subdomain. Currently this interface kernel is specific to electrons because the transport coefficients are assumed to be a function of the mean electron energy. A generic interface kernel with constant transport coefficients will have a much simpler Jacobian\\\hline
    InterfaceLogDiffusionElectrons & \(\displaystyle -\psi_{i,el} D_{k,n} e^{N_{k,n}} \nabla N_{k,n} \cdot \vec{n} / (l_{c,n}l_{c,el})\) & Used to include the diffusive flux of species $k$ into or out of a neighboring subdomain. Also currently specific to electrons.\\\hline
  \end{longtable}
\end{ThreePartTable}

\subsection{Zapdos Boundary Conditions}
\label{sec:zap_bcs}

Zapdos boundary conditions at the cathode are based on the work in \cite{hagelaar2000boundary} and \cite{sakiyama2007nonthermal}. For ions, electrons, and the electron energy, the most commonly used conditions are respectively (in strong form and without scaling factors):

\begin{equation}
  \vec{\Gamma_i}\cdot\vec{n} = \frac{1-r_i}{1+r_i}\left(\left(2a_i-1\right)\mu_i\vec{E}\cdot\vec{n}n_i + \frac{1}{2}v_{th,i}n_i\right)
  \label{eq:ion_bc_3}
\end{equation}
\begin{equation}
    \vec{\Gamma_e}\cdot\vec{n} = \frac{1-r_{dens}}{1+r_{dens}}\left(-\left(2a_e-1\right)\mu_e\vec{E}\cdot\vec{n}\left(n_e-n_{\gamma}\right) + \frac{1}{2}v_{th,e}\left(n_e-n_{\gamma}\right)\right) - \left(1-a_e\right)\gamma_p\vec{\Gamma_p}\cdot\vec{n}
  \label{eq:electron_bc_3}
\end{equation}
\begin{equation}
    \vec{\Gamma_{\epsilon}}\cdot\vec{n} = \frac{1-r_{en}}{1+r_{en}}\left(-\left(2a_e-1\right)\frac{5}{3}\mu_e\vec{E}\cdot\vec{n}\left(n_e\epsilon-n_{\gamma}\epsilon_{\gamma}\right) + \frac{5}{6}v_{th,e}\left(n_e\epsilon-n_{\gamma}\epsilon_{\gamma}\right)\right) - \frac{5}{3}\epsilon_{\gamma}\left(1-a_e\right)\gamma_p\vec{\Gamma_p}\cdot\vec{n}
  \label{eq:energy_bc_3}
\end{equation}

where $r_i$, $r_{dens}$, $r_{en}$ are the boundary reflection coefficients for ions, electrons, and electron energy respectively (more discussion on $r_{en}$ shortly), $\gamma_p$ is the secondary electron emission coefficient, $\epsilon_{\gamma}$ is the energy of the secondary electrons, $\vec{n}$ is the outward facing normal vector, and:

\begin{equation}
  a_k =
    \begin{cases}
      1, & sgn_k\mu_k\vec{E}\cdot\vec{n}>0 \\
      0, & sgn_k\mu_k\vec{E}\cdot\vec{n}\leq0
    \end{cases}
  \label{eq:a_3}
\end{equation}
\begin{equation}
  v_{th,k} = \sqrt{\frac{8T_k}{\pi m_k}}
  \label{eq:v_th_3}
\end{equation}
\begin{equation}
  n_{\gamma} = \left(1-a_e\right)\frac{\gamma_p\vec{\Gamma_p}\cdot\vec{n}}{\mu_e\vec{E}\cdot\vec{n}}
  \label{eq:n_gamma_3}
\end{equation}

where $v_{th,k}$ is the thermal velocity of species $k$ and $n_{\gamma}$ is the density of secondary electrons. A thermodynamic interfacial condition is also available:

\begin{equation}
  Hn_{e,g} = n_{e,l}
  \label{eq:electron_bc_thermo_3}
\end{equation}

For ions and electrons in the liquid phase, depending on the polarity of the discharge, a simple outflow BC is used at the counter electrode at the bottom of the liquid. Its strong form is:

\begin{equation}
  \vec{\Gamma_k}\cdot\vec{n} = -a_k \mu_k \sgn(q_k) e^{N_k} \nabla V \cdot \vec{n}
  \label{eq:outflow_3}
\end{equation}

For potential conditions, grounding is done using a DirichletBC class inherited from MOOSE. The other boundary condition incorporates an external ballast resistor:

\begin{equation}
  V_{source} + V_{cathode} = \left(e\vec{\Gamma_i} - e\vec{\Gamma_e}\right)AR
  \label{eq:cathode_3}
\end{equation}

A summary of the important boundary condition classes that Zapdos defines are summarized in \cref{tab:bcs}.

\begin{ThreePartTable}

  \begin{TableNotes}
  \end{TableNotes}

  \begin{longtable}{>{\centering}m{2in}| >{\centering}m{2in}| >{\raggedright\arraybackslash}m{2in}}
    \textbf{BoundaryCondition Name} & \textbf{Expression (strong form and without scaling factors)} & \textbf{Description}\\\hline\hline
    \endhead
    \caption{Important BoundaryConditions defined by Zapdos}
    \endfoot
    \caption{Important BoundaryConditions defined by Zapdos}
    \label{tab:bcs}
    \endlastfoot

    HagelaarIonAdvectionBC & First parenthetical term in \cref{eq:ion_bc_3} & Kinetic advective ion boundary condition\\\hline
    HagelaarIonDiffusionBC & Second parenthetical term in \cref{eq:ion_bc_3} & Kinetic diffusive ion boundary condition\\\hline
    HagelaarElectronBC & \cref{eq:electron_bc_3} & Kinetic electron boundary condition\\\hline
    HagelaarEnergyBC & \cref{eq:energy_bc_3} & Kinetic electron energy boundary condition\\\hline
    DCIonBC & \cref{eq:outflow_3} & Electric field driven outflow boundary condition\\\hline
    NeumannCircuitVoltageMoles\_KV & \cref{eq:cathode_3} & Circuit boundary condition for potential\\\hline
    MatchedValueLogBC & \cref{eq:electron_bc_thermo_3} & Henry's Law like thermodynamic boundary condition for specifying a specie concentration ratio at the gas-liquid interface\\\hline
  \end{longtable}
\end{ThreePartTable}

\subsection{Zapdos Materials}
\label{sec:zap_materials}

Transport, rate, and other properties are defined in class files in the materials directory of Zapdos. Gas properties (whether argon, air, etc.) are defined in the class file Gas.C. Aqueous solute properties are defined in the class file Water.C. In the gas phase, electron transport and Townsend rate coeffiecients are functions of the electron mean energy (or alternatively they can be functions of the local electric field). As shown in \cref{code:read_in}, this data is read in from a whitespace-delimited look-up table from a text input file. The look-up table data is parsed into interpolation objects. There are several interpolation types that MOOSE application developers can choose from; we have chosen to use a spline interpolator. Because a spline interpolator provides C$^2$ continuity, there are no derivative jumps with respect to the mean energy. This in turn makes for continuous Jacobian functions, leading to markedly improved convergence over a linear interpolator for example.

\begin{lstlisting}[language=C++, caption = Code for reading in electron transport and Townsend coefficient data from a lookup table in a text file, label = code:read_in]
  // Define path to look-up table
  std::string tdPath = "/src/materials/td_argon_mean_en.txt";
  std::string path = zapDir + tdPath;
  const char *charPath = path.c_str();

  // Create input file stream: myfile
  std::ifstream myfile (charPath);
  Real value;

  if (myfile.is_open())
  {
    // As long we haven't reached the end of file, read entries from the look-up table into respective data arrays
    while ( myfile >> value )
    {
      // Get mean energy values that Townsend and transport coefficients are a function of
      actual_mean_energy.push_back(value);
      myfile >> value;
      // Townsend ionization coefficient
      alpha.push_back(value);
      myfile >> value;
      // Townsend excitation coefficient
      alphaEx.push_back(value);
      myfile >> value;
      // Townsend elastic collision coefficient
      alphaEl.push_back(value);
      myfile >> value;
      // Electron mobility
      mu.push_back(value);
      myfile >> value;
      // Electron diffusivity
      diff.push_back(value);
    }
    myfile.close();
  }

  else std::cerr << "Unable to open file" << std::endl;

  // Create interpolation functions for Townsend and transport coefficients that depend on the mean energy
  _alpha_interpolation.setData(actual_mean_energy, alpha);
  _alphaEx_interpolation.setData(actual_mean_energy, alphaEx);
  _alphaEl_interpolation.setData(actual_mean_energy, alphaEl);
  _mu_interpolation.setData(actual_mean_energy, mu);
  _diff_interpolation.setData(actual_mean_energy, diff);
\end{lstlisting}

\Cref{code:mat_def} shows definition of both constant material properties and solution variable-dependent properties. In this particular codeblock the code tests whether the user wants the electron mobility and diffusivity to be functions of the mean energy. If so, the interpolator \textbf{sample} method is called to retrieve the property at the corresponding value for the mean energy (recall that $\epsilon$ is a function of $E_n$ and $N_e$; see \cref{eq:soln_vars}). In addition the interpolator's \textbf{sampleDerivative} method is also called; its returned value is used in the Jacobian methods of any kernels or boundary conditions that rely on the corresponding material property. If the user does not want to interpolate the transport coefficients, then they are set to some constant sane values. The code block also shows how the properties are scaled depending on the choice of potential units.

\begin{lstlisting}[language=C++, caption = Material property definition, label = code:mat_def]
  // Check whether user wants to interpolate transport coefficients as a function of mean energy, or just use constants
  if (_interp_trans_coeffs) {
    // Get value for mobility
    _muem[_qp] = _mu_interpolation.sample(std::exp(_mean_en[_qp]-_em[_qp])) * _voltage_scaling;
    // Get derivative of mobility with respect to the mean energy. Used in Jacobian computations
    _d_muem_d_actual_mean_en[_qp] = _mu_interpolation.sampleDerivative(std::exp(_mean_en[_qp]-_em[_qp])) * _voltage_scaling;
    // Get value for diffusivity
    _diffem[_qp] = _diff_interpolation.sample(std::exp(_mean_en[_qp]-_em[_qp]));
    // Get derivative of diffusivity with respect to the mean energy. Used in Jacobian computations
    _d_diffem_d_actual_mean_en[_qp] = _diff_interpolation.sampleDerivative(std::exp(_mean_en[_qp]-_em[_qp]));
  }
  else {
    // From bolos at atmospheric pressure and an EField of 2e5 V/m
    _muem[_qp] = 0.0352103411399 * _voltage_scaling; // units of m^2/(kV*s) if _voltage_scaling = 1000
    // No functional dependence on mean energy if transport coefficients are constant
    _d_muem_d_actual_mean_en[_qp] = 0.0;
    _diffem[_qp] = 0.297951680159;
    _d_diffem_d_actual_mean_en[_qp] = 0.0;
  }
\end{lstlisting}


\subsection{Meshing for Zapdos}
\label{sec:zap_meshing}

Meshes required for Zapdos input files are generated using the program Gmsh. \cite{geuzaine2009gmsh} An example Gmsh input file is shown in \cref{code:gmsh}. The scaling used in the mesh input file is the inverse of the scaling used in the Zapdos input file, e.g. $dom0Mult = 1/dom0Scale$. For typical plasma simulations, the characteristic length of the mesh is much finer in the boundary regions than in the plasma bulk. For the input file below, which is representative of the meshes used for \cref{sec:plasliq}, the characteristic length of the mesh is 2 nm at the cathode and 1 nm at the plasma-liquid interface with the mesh characteristic length peaking at 50 $\mu$m in the center of the discharge. In the liquid phase, the characteristic length in the bulk is 10 nm.

\begin{lstlisting}[language=C++, caption=Gmsh input file used to create plasma and liquid domains for simulations in \cref{sec:plasliq}, label=code:gmsh]
/* In this example we have chosen to scale the mesh to improve Jacobian conditioning */
dom0Mult = 1e3;
dom1Mult = 1e7;

/* We would comment the above two lines and uncomment the two lines below if we did not want to scale the mesh */
// dom0Mult = 1;
// dom1Mult = 1;

// Specify a 2 nm characteristic length at the left edge of the plasma
Point(1) = {0, 0, 0, 2e-9 * dom0Mult};

// 50 micron characteristic length in plasma center
Point(3) = {.5e-3 * dom0Mult, 0, 0, 50e-6 * dom0Mult};
Line(2) = {1,3};

// 1 nm characteristic at gas-liquid interface
Point(8) = {1e-3 * dom0Mult, 0, 0, 1e-9 * dom0Mult};
Line(7) = {3,8};

// 10 nm characteristics at liquid domain center and right edge
Point(9) = {1e-3 * dom0Mult + 50e-9 * dom1Mult, 0, 0, 10e-9 * dom1Mult};
Line(8) = {8,9};
Point(10) = {1e-3 * dom0Mult + 100e-9 * dom1Mult, 0, 0, 10e-9 * dom1Mult};
Line(9) = {9,10};

// Create physical mesh objects that will be recognized by Zapdos
Physical Line(0) = {2,7};
Physical Line(1) = {8,9};
\end{lstlisting}

\subsection{Postprocessing Zapdos results}
\label{sec:zap_post}

Zapdos by default outputs results to an exodus file. Our preferred tool for processing these results is the open-source application Paraview. \cite{paraview} Paraview offers both a GUI as well as python modules for processing data in a variety of formats, including exodus. Using the python interface, we can script creation of Paraview readers and write data for specific time steps as well as perform many other functions. In the case where we are simulating to steady-state a DC discharge from some arbitrary initial state, solution data from the last time point are mostly what we care about. Using Paraview's CSV writer from within python, we write the data to a csv file, from which we can then load the data into numpy data arrays. These steps are achieved in a single python method that we developed; it is shown in \cref{code:load_data}. Once nodal and elemental variables have been loaded into numpy arrays, python's matplotlib package can be used to render publication-quality figures. This is automated using generic figure scripts shown in \cref{code:elem_plot,code:nodal_plot}.


\section{Modifying the MOOSE Framework}
\label{sec:moose}

Essential to the simulation of plasma-liquid systems is the ability to couple the physics of the two domains across their interface. Somewhat surprisingly for a framework used for very mature multi-physics application codes, when first starting to investigate simulation of plasma-liquids, MOOSE lacked a straightforward way to interface physics across domains. E.g. one could not set the flux of a specie A on the domain 0 side of an interface equal to the flux of a specie B on the domain 1 side of the interface. If one wanted to ensure continuity of a species flux across an interface, he had no choice but to use the same variable on both domains, which has the unfortunate side-effect for a Continuous Galerkin formulation of also requiring continuity of the species concentration across the interface as well. This condition is of course physically unrealistic in almost all cases; as two examples, hydrophilic H$_2$O$_2$ has six orders of magnitude higher concentration on the liquid side of an interface and hydrophobic NO has two orders of magnitude higher concentration on the gas side of the interface. Thus in order to solve the plasma-liquid equations in a code built on top of MOOSE, the MOOSE framework itself had to be altered to allow creation of residual and Jacobian statements for interfacial conditions like continuity of flux.

MOOSE has several ``systems''; before modification by the author, the systems directly responsible for computing residual contributions from pieces of user governing equations were Kernels (including a few closely related variants like Nodal Kernels and Dirac Kernels), DGKernels (for discontinuous Galerkin computations), constraints, and boundary conditions. The MOOSE constraints system could potentially have been used in a somewhat inelegant way to impose interfacial conditions; however, this would have required at a minimum purchase of an expensive and proprietary commercial meshing package. Instead the author chose to code a brand new MOOSE system called ``Interface Kernels.'' At the time of writing, the implementation of Interface Kernels has required modification or creation of 40 framework files, encompassing the addition or modification of 1,400 lines of code.

As described in \cref{sec:zapdos}, at the core of MOOSE and any application like Zapdos built on top of it is the computation of residual and Jacobian statements (see \cref{eq:Newton}). Adding a new system requires supplying all the code necessary to route the MOOSE framework core residual and Jacobian compute threads to the user provided residual and Jacobian statements. In order to add the new InterfaceKernel system, intelligent decisions about the user interface had to be made. InterfaceKernels resemble most strongly a cross between DGKernels and Integrated Boundary Conditions (IntegratedBCs). DGKernels operate on internal sides of subdomains, providing the user access to solution variable values and gradients at surface interfaces between elements; IntegratedBC's require the user to provide a boundary (or boundaries) to restrict the condition to. InterfaceKernels make use of both of these features: the InterfaceKernel inherits from the DGKernel class, providing InterfaceKernel objects with the ability to access variables on either side of element interfaces, and the BoundaryRestrictable class, allowing the user to specify the internal surfaces on the mesh where the interfacial conditions will live.

After deciding on the user interface for the InterfaceKernel class, the author had to program the MOOSE core residual and Jacobian compute threads to call the respective InterfaceKernel member functions. The partial stack trace shown in \cref{code:stack} shows the architecture for how InterfaceKernel residuals get called. Full stack traces trace function calls from the main program to the function under investigation; they are very useful for determining how large programs like MOOSE are structured. In \cref{code:stack} we show the parts of the stack trace that are important for the newly implemented interface kernel system. We describe each function in the stack trace with a comment; at the end of each comment we indicate whether the function described is newly implemented by us or whether it already existed in the framework. Shown as item \# 4 in \cref{code:stack}, the ThreadedElementLoopBase::operator method is used to loop over elements; it can call both residual and Jacobian threads. \Cref{code:ThreadedElementLoopBase} shows the calls to different geometric objects. The call to onElement computes kernel methods that exist in element volumes; onBoundary calls integrated boundary condition methods that exist on the external sides of the domain; onInternalSide calls discontinuous Galerkin kernels that exist on element sides internal to the domain; finally, the last object call, onInterface, is the call implemented by the author that calls methods for element sides that lie along an interface between subdomains. Both the ComputeResidualThread and ComputeJacobianThread classes are children of the ThreadedElementLoopBase class and must implement the onInterface method that is called by their parent.

\begin{lstlisting}[language=C++, caption = Stack trace showing the architecture for how InterfaceKernel residuals get called, label = code:stack]
    /* Call residual functions specific to application physics. In this case we are calling InterfaceAdvection which takes the advective flow of electrons out of the gas domain and creates an advective flow of electrons into the liquid domain. (New capability) */
#0  InterfaceAdvection::computeQpResidual (this=0xf72bf0, type=Moose::Element) at /home/lindsayad/projects/zapdos/src/interfacekernels/InterfaceAdvection.C:56

    /* This function tests to see whether we are on the master or slave side of the interface. If we are on the master side we use the test functions associated with the master side and visa versa for the slave side. This function is also responsible for adding the residual to the correct residual block, e.g. if we are on the master side, the residual should be added to the residual block for the master variable. This function calls child classes computeQpResidual functions like that of InterfaceAdvection above (New capability) */
#1  0x00007ffff69beae9 in InterfaceKernel::computeElemNeighResidual (this=0xf72bf0, type=Moose::Element) at /home/lindsayad/moose/framework/src/interfacekernels/InterfaceKernel.C:57

    /* A simple function that calls InterfaceKernel::computeElemNeighResidual twice. The first time it tells the InterfaceKernel class to compute the master side residual. The second time it tells InterfaceKernel to compute the slave side residual (Pre-existing capability) */
#2  0x00007ffff64463b6 in DGKernel::computeResidual (this=0xf72bf0) at /home/lindsayad/moose/framework/src/dgkernels/DGKernel.C:134

    /* Function that sweeps through all existing InterfaceKernel child objects and calls their compute residual threads; e.g. example thread is ComputeResidualThread::onInterface -> DGKernel::computeResidual -> InterfaceKernel::computeElemNeighResidual -> InterfaceAdvection::computeQpResidual (New capability) */
#3  0x00007ffff67dbd69 in ComputeResidualThread::onInterface (this=0x7fffffffc810, elem=0xd3c570, side=0, bnd_id=2) at /home/lindsayad/moose/framework/src/base/ComputeResidualThread.C:162

    /* Function that iterates through both residual threads like the thread described immediately above and Jacobian threads like ComputeJacobianThread::onInterface -> ComputeFullJacobianThread::computeInternalInterFaceJacobian -> InterfaceKernel::computeOffDiagJacobian -> InterfaceKernel::computeOffDiagElemNeighJacobian -> InterfaceAdvection::computeQpOffDiagJacobian (Some new capability) */
#4  0x00007ffff64b8903 in ThreadedElementLoopBase<libMesh::StoredRange<libMesh::MeshBase::const_element_iterator, libMesh::Elem const*> >::operator() (this=0x7fffffffc810, range=..., bypass_threading=false) at /home/lindsayad/moose/framework/include/base/ThreadedElementLoopBase.h:180
\end{lstlisting}

\begin{lstlisting}[language = C++, caption = Snapshot of different geometric object calls in ThreadedElementLoopBase::operator, label = code:ThreadedElementLoopBase]
      // Call residual and Jacobian functions of objects associated with volumetric elements. These are Kernel objects
      onElement(elem);

      for (unsigned int side=0; side<elem->n_sides(); side++)
      {
        // Get IDs of mesh boundaries where boundary conditions are defined
        std::vector<BoundaryID> boundary_ids = _mesh.getBoundaryIDs(elem, side);

        if (boundary_ids.size() > 0)
          // Loop over boundary IDs
          for (std::vector<BoundaryID>::iterator it = boundary_ids.begin(); it != boundary_ids.end(); ++it)
            // Call residual and Jacobian functions associated with boundaries. These are IntegratedBC objects
            onBoundary(elem, side, *it);

        if (elem->neighbor(side) != NULL)
        {
          // Call residual and Jacobian functions associated with mesh internal sides. These are DGKernel objects
          onInternalSide(elem, side);
          if (boundary_ids.size() > 0)
            for (std::vector<BoundaryID>::iterator it = boundary_ids.begin(); it != boundary_ids.end(); ++it)
              // Call residual and Jacobian functions associated with interfaces between subdomains. These are the newly implemented InterfaceKernel objects
              onInterface(elem, side, *it);
        }
      } // sides
\end{lstlisting}

\Cref{code:ComputeResidualThread} shows the implementation of the onInterface method in the ComputeResidualThread class. The method takes as arguments the current element, one of the element's sides, and a boundary ID (bnd\_id). The method first checks whether any \_interface\_kernels exist and are active on the boundary specified by bnd\_id. The initialization of \_interface\_kernels will be discussed later. The element's neighbor is accessed using its neighbor method. Currently, the interface kernel system does not support mesh adaptivity; this is checked by comparing neighbor->level() and elem->level() (the level method returns the level of element refinement). After checking whether the neighboring element is active (relevant for transient simulations), a reinitialization of the element face and neighboring face materials is performed. In the most important lines of the method, all of the \_interface\_kernels active on bnd\_id are iterated over, and their individual computeResidual methods are called. The logic for the onInterface method implemented in the ComputeJacobianThread class is very similar to that of ComputeResidualThread.

\begin{lstlisting}[language = C++, caption = ComputeResidualThread::onInterface method. The logic is much the same for the ComputeJacobianThread::onInterface method. , label = code:ComputeResidualThread]
void
ComputeResidualThread::onInterface(const Elem *elem, unsigned int side, BoundaryID bnd_id)
{
  // Check whether any interface kernels are active on the provided boundary ID
  if (_interface_kernels.hasActiveBoundaryObjects(bnd_id, _tid))
  {

    // Pointer to the neighbor we are currently working on.
    const Elem * neighbor = elem->neighbor(side);

    if (!(neighbor->level() == elem->level()))
      mooseError("Sorry, interface kernels do not work with mesh adaptivity");

    // Check whether neighboring element is active. E.g. some transient simulations may not simulate all of the subdomains for all time steps
    if (neighbor->active())
    {
      _fe_problem.reinitNeighbor(elem, side, _tid);

      // Make sure material properties are up-to-date
      _fe_problem.reinitMaterialsFace(elem->subdomain_id(), _tid);
      _fe_problem.reinitMaterialsNeighbor(neighbor->subdomain_id(), _tid);

      const std::vector<MooseSharedPointer<InterfaceKernel> > & int_ks = _interface_kernels.getActiveBoundaryObjects(bnd_id, _tid);
      // Iterate over all interface kernels active on the provided boundary ID and compute the corresponding residuals
      for (std::vector<MooseSharedPointer<InterfaceKernel> >::const_iterator it = int_ks.begin(); it != int_ks.end(); ++it)
        (*it)->computeResidual();

      _fe_problem.swapBackMaterialsFace(_tid);
      _fe_problem.swapBackMaterialsNeighbor(_tid);

      {
        Threads::spin_mutex::scoped_lock lock(Threads::spin_mtx);
        _fe_problem.addResidualNeighbor(_tid);
      }
    }
  }
}
\end{lstlisting}

The computeResidual method called from ComputeResidualThread :: onInterface is inherited from the DGKernel class. It calls in succession InterfaceKernel :: computeElemNeighResidual(Moose::Element) and InterfacKernel :: computeElemNeighResidual(Moose::Neighbor). The InterfaceKernel :: computeElemNeighResidual method is shown below in \cref{code:InterfaceKernel}. Depending on whether Moose::Element or Moose::Neighbor is passed as the method argument, the space of test functions is taken from the focused element or the neighboring element respectively. The correct residual block, either for \_var in the focused element or \_neighbor\_var in the neighboring element, is similarly determined from the method argument. At the end of the method, we loop over all the quadrature points in the element and add \_JxW[\_qp] * \_coord[\_qp] * computeQpResidual(type) to the i$^{th}$ residual. Here \_JxW represents the quadrature weight, \_coord is a scaling factor for converting from Cartesian to other coordinate systems (e.g. cylindrical or spherical), and computeQpResidual is the residual computed at the quadrature points by the current InterfaceKernel child class.

\begin{lstlisting}[language = C++, caption = The InterfaceKernel :: computeElemNeighResidual method responsible for calling compueQpResidual methods implemented in the various children of the InterfaceKernel class, label = code:InterfaceKernel]
void
InterfaceKernel::computeElemNeighResidual(Moose::DGResidualType type)
{
  bool is_elem;
  // If type == Moose::Element, we are on the master side of the interface, which through MOOSE convention we also call the ``element'' side
  if (type == Moose::Element)
    is_elem = true;
  else
    is_elem = false;

  // Get the test functions matching the side of the interface we are on
  const VariableTestValue & test_space = is_elem ? _test : _test_neighbor;

  // Make sure residual is going towards the correct variable
  DenseVector<Number> & re = is_elem ? _assembly.residualBlock(_var.number()) :
                                       _assembly.residualBlockNeighbor(_neighbor_var.number());

  // Loop over quadrature points and test functions and add residual contributions
  for (_qp = 0; _qp < _qrule->n_points(); _qp++)
    for (_i = 0; _i < test_space.size(); _i++)
      re(_i) += _JxW[_qp] * _coord[_qp] * computeQpResidual(type);

}
\end{lstlisting}

An example of a computeQpResidual method implemented in a child class of InterfaceKernel is taken from the InterfaceAdvection class defined in Zapdos. It is shown in \cref{code:computeQpResidual}. The InterfaceAdvection class ensures that all the species being advected from one subdomain flow into the neighboring subdomain. It represents an interfacial condition acting only on the variable living on the focused element; as can be seen from the switch and case logic, no residual contribution is given for the variable living on the neighboring element. Although not relevant for describing the InterfaceKernel system, note that \_r\_units is a data member controlled by the user enabling mesh scaling and improved Jacobian conditioning. As noted previously, species concentration variables are in a logarithmic form such that std::exp(\_neighbor\_value) actually represents the physical concentration of the neighboring specie; \_mu\_neighbor is the mobility of the neighboring species, \_sgn\_neighbor is the charge sign of the neighboring speices, and \_grad\_potential\_neighbor is the gradient of the potential on the neighboring side of the interface.

\begin{lstlisting}[language = C++, caption = InterfaceAdvection::computeQpResidual method, label = code:computeQpResidual]
Real
InterfaceAdvection::computeQpResidual(Moose::DGResidualType type)
{
  Real r = 0;

  switch (type)
  {
  case Moose::Element:
    // Add the flux of electrons from the neighboring subdomain to the balance equation for electrons in the current subdomain
    r = _mu_neighbor[_qp] * _sgn_neighbor[_qp] * -_grad_potential_neighbor[_qp] * _r_neighbor_units * std::exp(_neighbor_value[_qp]) * _normals[_qp] * _test[_i][_qp] * _r_units;
    break;

  case Moose::Neighbor:
    // This condition is only imposed on the master side of the interface, thus we do not add any residual contribution to the neighboring side
    r = 0.;
    break;
  }

  return r;
}
\end{lstlisting}

Interface kernels are read from their own input block in a MOOSE application's input file. An example is shown in \cref{code:interface_input}. The act method in the AddInterfaceKernelAction class calls FEProblem :: addInterfacerKernel which in turn calls NonlinearSystem :: addInterfaceKernel. NonlinearSystem :: addInterfaceKernel adds the interface kernels from the input file to the protected \_interface\_kernels data member which are then accessible to the ComputeResidualThread and ComputeJacobianThread classes through the public accessor method, getInterfaceKernelWarehouse. The details of this initialization process can be found in the source code at \cite{mooseSite}. An important thing to note about interface kernels is that they are uniquely assigned to elements on one side of the interface. Without unique assignment, there could be no organized residual definitions like that shown in \cref{code:computeQpResidual}. Unique assignment is achieved by using libMesh's sideset objects. The block used to create the sideset 'master1\_interface' that is then used to uniquely define the interface kernel of \cref{code:interface_input} is shown in \cref{code:sideset}. Using the built-in SideSetsBetweenSubdomains class, the new sideset is constructed on the block 1 side of the interface.

\begin{lstlisting}[language=C++, caption = Example of input block for an interface kernel (InterfaceAdvection in this case), label = code:interface_input]
[InterfaceKernels]
  // This condition adds the advective flux of electrons coming from the gas phase to the balance equation of electrons in the liquid phase
  [./em_advection]
    type = InterfaceAdvection
    // ``mean_en''  is the gas phase electron energy. There is no electron energy variable in the liquid phase
    mean_en_neighbor = mean_en
    // The ``potential'' variable spans both gas and liquid phases since it is continuous at the interface
    potential_neighbor = potential
    // ``em'' is the gas phase electron density. It is the slave variable corresponding to the ``emliq'' master variable
    neighbor_var = em
    // ``emliq'' is the liquid phase electron density and is the master variable for this InterfaceAdvection object
    variable = emliq
    // This interfacial condition is imposed on the liquid side of the interface. ``1'' denotes the liquid subdomain. ``0'' denotes the plasma subdomain
    boundary = master1_interface
    // Following two lines specify the mesh scaling for both liquid and plasma subdomains respectively
    position_units = ${dom1Scale}
    neighbor_position_units = ${dom0Scale}
  [../]
[]
\end{lstlisting}

\begin{lstlisting}[language = C++, caption = Example of how to create a sideset\, in this case 'master1\_interface'\, that can then be used in definition of an interface kernel, label = code:sideset]
[MeshModifiers]
  [./interface_again]
    type = SideSetsBetweenSubdomains
    // ``1'' denotes the liquid subdomain.
    master_block = '1'
    // ``0'' denotes the plasma subdomain.
    paired_block = '0'
    // This new boundary will be a sideset tied to the liquid subdomain
    new_boundary = 'master1_interface'
  [../]
[]
\end{lstlisting}

Addition of the interface kernel system to the MOOSE framework enables the interfacing of plasma and liquid domains required to obtain the results in \cref{sec:plasliq}. Moreover, the system should be applicable to many other scientific and engineering applications that inherit from the MOOSE framework. In the words of MOOSE founder Derek Gaston: ``Thanks for all this work! To my best knowledge I think this is the first time an external contributor has added a whole new "System" in MOOSE! Definitely a landmark occasion! This is a good one too... LOTS of people will use this for many years to come (including myself!).''
