\chapter{BACKGROUND \& PURPOSE}
\label{chap-one}

\section{Basic Science of Plasma-Liquid Interactions}
\label{sec:intro_plasliq}

There is a general interest in the study of plasma-liquid interactions within the
scientific community for an array of applications, including but not limited to biomedicine and biological disinfection \cite{Kong2009b,Laroussi2009,Shimizu2014c,VonWoedtke2014a,VonWoedtke2013a,Joubert2013a}, chemical disinfection \cite{Johnson2006,Locke2006,Theron2008}, and agricultural applications. \cite{Park2013b,Lindsay2014} In order to successfully realize these applications and develop mature technologies, the basic science underlying these coupled plasma gas-liquid systems must be well understood. Recent experimental,\cite{Lukes2014b,Bruggeman2009d,Pavlovich2013g,Traylor2011h,yagi2015two,bruggeman2008dc} modeling,\cite{Babaeva2014b,Tian2014,Chen2014a}, and review works \cite{locke2011review,bruggeman2009non} have revealed many of the important physiochemical processes. However, more work remains, especially in the understanding of multi-phase convective systems like jets or streamers over water.

This paper addresses several points regarding transport in plasma-liquid systems. First, convection-induced evaporation of water can lead to rather large temperature gradients at the gas-liquid interface, as large as 8 Kelvin/100 microns in the gas boundary layer. The drop in temperature from gas to liquid due to evaporative cooling has not yet been considered in the plasma literature; it is directly relevant to plasma chemistry because of the Arrhenius dependence of many reaction rate constants on temperature. In addition to evaporative cooling, convection leads to depletion of water vapor concentration in the vicinity of the discharge. This will also impact gas-phase plasma chemical reactions that depend on water vapor as a reactant. \cite{winter2013feed}

This study illustrates the sharp drop in the concentrations of highly reactive species from the liquid surface to the liquid bulk. In the liquid near the interface, OH concentrations fall by as many as 9 orders of magnitude over tens of microns. Though the existence of strong gradients in aqueous radical concentrations is well known in the biochemistry literature, \cite{Bachi2013,Halliwell}, it has yet to receive significant attention in the plasma-liquid and plasma-medicine communities. Reference \cite{Chen2014a} explicitly models the limited penetration of primary plasma-generated species into aqueous solutions. Recognizing these strong gradients in reactive radical concentrations has important implications for understanding plasma therapeutics in which cellular effects are often observed a significant distance (1 cm in case of subcutaneous tumors \cite{Graves2014review}) away from local plasma treatment. The results shown here are consistent with the model developed by Graves \cite{Graves2014review} where short-lived plasma-generated radicals from the gas phase react in the surface layer of the aqueous/biological phase to form longer-lived species such as oxidized and/or nitrated/nitrosylated proteins, peptides, lipids which can then enter or communicate with surface cells.
