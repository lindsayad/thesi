\chapter{INTRODUCTION \& BACKGROUND}
\label{chap:intro}

\section{Basic Science of Plasma-Liquid Interactions}
\label{sec:intro_plasliq}

There is a general interest in the study of plasma-liquid interactions within the scientific community for an array of applications, including but not limited to biomedicine and biological disinfection \cite{Kong2009b,Laroussi2009,Shimizu2014c,VonWoedtke2014a,VonWoedtke2013a,Joubert2013a}, chemical disinfection \cite{Johnson2006,Locke2006,Theron2008}, and agricultural applications. \cite{Park2013b,Lindsay2014} In order to successfully realize these applications and develop mature technologies, the basic science underlying these coupled plasma gas-liquid systems must be well understood. The richness and complexity of plasma-liquid systems require detailed study; an appreciation for the many physics involved can be gained by examining \cref{fig:central}. Recent experimental, modeling, and review works have enhanced our understanding of many of the important physiochemical processes involved in these systems. Lukes et. al. \cite{Lukes2014b} conducted an in-depth study of the aqueous phase chemistry produced by atmospheric pressure air discharges. Through the use of phenol as a chemical probe, the group saw evidence of generation of OH, NO, and NO$_2$ radicals at the interface as well as long term generation of OH and NO$_2$ through dissociation of ONOOH, itself a product of the reaction of HNO$_2$ and H$_2$O$_2$. Lukes's elucidation of the radical generation pathway from peroxynitrite decay corroborates the work conducted by Traylor et. al. \cite{Traylor2011h} in which the long-term bacterial efficacy of plasma-activated water (PAW) was investigated. Traylor found that hydrogen peroxide and nitrite concentrations diminished significantly over the course of several days, corresponding to a significant decrease in the solution's bactericidal properties. The decay in H$_2$O$_2$ and NO$_2^-$ concentrations is consistent with Lukes's reaction pathway of $H_2O_2 + H^+ + NO_2^- \rightarrow ONOOH$; the drop in the solution's anti-microbial behavior likely corresponds to a drop in OH and NO$_2$ radical production through ONOOH decay.

\begin{figure}[htbp]
  \centering
  \includegraphics[width=.9\textwidth]{central_thesis_figure_again.png}
  \caption{Cartoon of plasma-liquid experiment with identification of key variables of interest.}
  \label{fig:central}
\end{figure}

Bruggeman et. al. \cite{Bruggeman2009d} investigated DC discharges generated directly in liquid. They observed two distinct modes: for low aqueous solution conductivities a streamer-like discharge formed in the liquid itself; at higher conductivities the discharge was generated in a large gaseous bubble. For both modes the group was able to measure the electron density and gas temperature of the discharges. Pavlovich et. al. \cite{Pavlovich2013g} investigated regime changes in gas chemistry when treating E. coli. They discovered that at low power densities, the dominant reactive specie is ozone; at high power densities the gas phase chemistry becomes NO$_x$ dominated. The researchers also observed that in the low power density regime, the gas has to undergo substantial mixing with the liquid phase in order to kill the bacteria infesting the solution. They hypothesized that this is because of the relatively high hydrophobicity of ozone; diffusion alone does not generate sufficient mass transfer of ozone between gas and liquid phases. Yagi and co-authors investigated the impact of varying gas flow rate on humidity and OH using two-dimensional laser induced fluorescence (LIF). \cite{yagi2015two} They found that increasing gas flow rate creates a low-humidity region in the vicinity of the discharge, which in turn leads to a decreased rate of OH radical production. They reason that this change in gas phase composition will likely affect radical fluxes to the liquid surface. In another work by Bruggeman \cite{bruggeman2008dc}, the behavior of point-to-plane DC discharges impinging on water is explored. When water is the cathode, the discharge is filamentary in nature; when the water is the anode, the discharge is much more stable and diffuse. In the latter configuration, the rotational temperatures of OH and N$_2$, good indicators of the gas temperature, are identically 3250 K in the discharge's positive column; however, near the water anode, the temperature drops by 2500 K. The authors hypothesize that in the case of the water anode, the water acts as both an electrical stabilizer and a heat sink.

Each of the experimental works mentioned above describes particular aspects of the coupled plasma-liquid system. Some discuss the aqueous chemistry generated by the plasma. Others explore the electrical properties of the discharge or probe the gas phase chemistry using optical diagnostics. However, what is really required for a thorough understanding of coupled plasma-liquid systems is a model capable of describing the unity of all these experimental observations. Such a model must be able to describle phenomena that occur on vastly different time and length scales. For instance electron transport occurs on nanosecond time scales whereas some reactions in aqueous solution take place over the course of days as witnessed by the observations in \cite{Traylor2011h}. Some modeling works have recently been conducted that begin to realize the comprehensive plasma-liquid description we desire. One excellent work, \cite{Chen2014a}, uses three decoupled regions to explore the plasma-liquid dynamics: a bulk gas region, a gas-liquid interface layer, and a semi-infinite liquid region. One of the key conclusions of that work is that highly reactive plasma chemistry (represented by OH$_2$ or O$_3$) only penetrates about 10-20 $\mu$m into the liquid bulk. With He-O$_2$ as the working gas and using a low power density, they predict that dry downstream chemistry will be dominated by O, O$_2$(a) and O$_3$. They predict that liquid phase chemistry will depend principally on superoxide (O$_2^-$), H$_2$O$_2$, and either HO$_2$ or O$_3$, with the latter two species decaying in the first tens of microns while the former two persist for milimeter scales. While the work gives a comprehensive description of the discharge conditions and the chemistries in the respective gas and liquid phases, the decoupled nature of the three domains makes it a non-ideal framework for investigating full coupling between the phases. For instance, particle fluxes are assumed to be mono-directional from the discharge phase to the gas-liquid interface layer. Thus evaporation of species and its effect on discharge physics and chemistry are not considered. Similarly, the effect of water evaporation on both gas and liquid temperature profiles and subsequently on reaction rates are not considered. Many of these effects can only be realized with a fully-coupled, bidirectional model.

Some of the most detailed plasma-liquid modeling work has come out of Mark Kushner's group at the University of Michigan. A particularly seminal work is that of Tian et. al. \cite{Tian2014}, wherein they report on results obtained using the model \textit{nonPDPSIM}. More detailed descriptions of \textit{nonPDPSIM} can be found in \cite{lay2002breakdown,xiong2010surface}. The model includes solution of Poisson's equation, (drift)-diffusion equations for (charged) and neutral species, an electron energy equation, and a radiation transport equation. Using this highly detailed description of the physics, the authors are able to make predictions about the important species formed in both gas and liquid phases as well as the mechanisms by which they are formed. For instance, aqueous O$_3$ comes primarily from dissolution of O$_3$(g) whereas significant portions of the OH(aq) and H(aq) concentrations can be attributed to photionization and photodissociation of H$_2$O at the liquid surface. Reactivity at a substrate positioned a few hundred microns into the liquid phase comes primarily from H$_2$O$_2$, O$_3$, and ONOO$^-$, all species that have received significant attention in the experimental plasma-liquid work. The work in \cite{Babaeva2014b} extends \textit{nonPDPSIM} to invesigation of how cells and tissue below a water layer might distort and affect the electric fields and particles fluxes coming form the plasma. Though \textit{nonPDPSIM} is perhaps the finest code currently being used in the plasma-liquid community, it does have some limitations. As described in \cite{xiong2010surface} much of the physics is segregated, e.g. charged particle densities are updated before updating the electron energy which is in turn updated before neutral particle densities, etc. While in many simulation cases segregated methods can be more efficient in terms of memory usage (and sometimes overall solution time), fully coupled methods are required when physics are very tightly coupled. In the case of very tight coupling, segregated methods may not converge. Such tight coupling may be realized when charged and neutral species transport, heat and momentum transport, radiation transport, and Maxwell's equations are all considered simultaneously.

Building off the work of \cite{morrow2011time}, Shirafuji et. al. investigated electric double layer formation in an arbitrary liquid medium XY in contact with an RF discharge. \cite{shirafuji2014numerical} In their simulation, implemented in the proprietary package Comsol, the bottom of the liquid phase is grounded while the potential at the other end of the domain oscillates. When the powered electrode is at a positive potential and ions are being pushed from the gas phase into the liquid phase, the authors observe a postive charge layer on the liquid side of the interface and a small negative charge layer at the bottom of the liquid volume. When the powered electrode is negative, a negative charge layer forms on the liquid side of the interface with a small positive charge layer at the liquid bottom. The authors conclude by suggesting that less mobile ions may preferentially appear at the interface.

Despite many of the excellent modeling works already published, numerous basic science questions about plasma-liquid systems remain. Several important questions are enumerated below:

\begin{enumerate}
\item \label{itm:advection} None of the comprehensive models described in the literature consider the role of convective fluid flow in the plasma-liquid dynamics. How does advection in atmospheric jets or corona discharges affect temperature profiles and reaction kinetics, the rate of mass transfer from gas to liquid, or the distribution of species in both gas and liquid phases?
\item \label{itm:interface} How do assumptions about conditions at the interface feed back into the plasma and liquid dynamics? For example, modellers \cite{Tian2014,shirafuji2014numerical} generally assume that electrons moving from gas to liquid have a sticking coefficient of unity while there is no experimental evidence to either disprove or support that assumption. How would changing electron and/or energy absorption at the interface affect plasma dyanmics and liquid characteristics?
\item \label{itm:spreading} In a work investigating solvation of electrons, Rumbach et. al. \cite{rumbach2015solvation} observed spreading of a needle-to-water discharge as solution conductivity decreased. What is the mechanism that explains this spreading?
\end{enumerate}

The work presented in this dissertation does much to answer the questions raised in \cref{itm:advection,itm:interface} and in the process lays a roadmap for answering \cref{itm:spreading}. The questions raised in \cref{itm:advection}, while unique to the low-temperature plasma community, fall largely under the umbrella of traditional chemical engineering transport and consequently can be answered using traditional chemical engineering software such as Comsol. \cite{comsolSite} However, \cref{itm:interface,itm:spreading} require a thorough description of electron and electron energy transport in the plasma. In the author's personal experience, Comsol is highly inefficient and may even fail to converge when solving the highly non-linear equations that describe plasmas. The problem is not impossible, as evidenced by work in \cite{shirafuji2014numerical,sakiyama2007nonthermal,sakiyama2006finite}; however, in all the works referenced the number of degrees of freedom is relatively small, thus an inefficient solve may be tolerable (to patient people). As problems get larger, however, inefficient solves become intolerable. It becomes necessary to have control over definition of Jacobian functions to ensure the most efficient solutions of non-linear equations. Moreover, in order to reduce computation time, it may be highly desirable to deploy the simulation across many CPU cores. This may be cost prohibitive when using a commercial simulation package. Recognizing the need for a multi-physics framework code and preferring not to start from scratch when many codes already exist, we perused OpenFOAM\cite{foamSite}, Elmer\cite{elmerSite}, and Code Aster\cite{asterSite} before settling on the Multiphysics Object-Oriented Simulation Environment (MOOSE) primarily developed by Idaho National Lab (INL).\cite{mooseSite} MOOSE is a finite element framework that pushes coding of the physical governing equations onto an application developer; it is the responsibility of the developer to write residual and Jacobian functions that define the physics and dictate the efficiency of the non-linear solve respectively. We have created a MOOSE application called Zapdos, whose source code can be found at \cite{zapdosSite}, for simulation of low-temperature plasmas with the possiblity for coupling to liquid systems. For those familiar with commercial multi-physics packages like Comsol, MOOSE can be thought of as being analogous to the base Comsol Multiphysics framework, while Zapdos is somewhat analogous to Comsol's Plasma Module. Though Zapdos is stacked on top of MOOSE, its construction has been an intensive task. To date Zapdos has over 23,000 lines of code; that number does not include the 1,400 lines that we added to the MOOSE framework itself necessary to enable interfacing of plasma and liquid domains.

To summarize, having reviewed the literature and having identified some key fundamental questions remaining in the field of plasma-liquids, it is our opinion that a new tool for plasma-liquid simulation, Zapdos, is necessary for the following reasons:

\begin{itemize}
\item \label{itm:parallel} Coupling of plasmas and liquids is a rich area of physics and chemistry that involves ionized gas dynamics, heat, momentum, and mass transport of both charged and neutral species, behavior of charges at interfaces, electrochemistry, etc. Tens to hundreds of species can be involved in hundreds to tens of thousands of reactions. This is a multi-physics problem with the potential for hundreds of thousands to millions of degrees of freedom. Consequently a versatile simulation framework should be massively parallelizable. Any applications built on top of MOOSE are automatically parallel; some applications have run on over 100,000 CPU cores. MOOSE is of particular value to our group because it is certified to run on ORNL's Titan, for which we have designated computing time allotments.
\item \label{itm:implicit} By default, MOOSE applications are fully implicit and fully coupled. Full coupling enables solution of tightly coupled, highly-nonlinear physics problems such as those encountered in plasma-liquid systems. Fully implicit schemes allow stable simulation of physics that occur over dramatically different time scales.
\item \label{itm:extensible} Because of its open source nature, MOOSE is highly flexible and highly extensible. As will be described later in \cref{sec:moose}, a competent programmer can add capabilities to the framework that are necessary to his application's success. Moreover, MOOSE's user object system allows wrapping of external libraries. This has the potential to be extremely useful for some future work; in \cref{sec:plasliq} we conclude from our results that atomistic or molecular simulations at the interface may be necessary for elucidation of some key interfacial properties. Other MOOSE researchers are already investigating wrapping SPPARKS (a kinetic Monte Carlo simulator) and LAMMPS (a molecular dynamics simulator) for their own applications. We may do something similar.
\item \label{itm:math} In the same vein as the above bullet, MOOSE employs perhaps the most general technique for discretizing partial differential equations, the Finite Element Method (FEM). By default MOOSE applications use a Continuous Galerkin discretization, but the application developer can easily add Discontinuous Galerkin methods. Additionally, MOOSE offers a wide array of test and shape functions, allowing the developer to, for example, reproduce finite volume methods.
\item \label{itm:jacobian} MOOSE gives the application developer complete control over residual and Jacobian function definitions. Consequently, any physics that can be written in a weak form can be included in a MOOSE application.
\item \label{itm:open_source} In the era of a global internet, it is our firm belief that codes used for academic purposes should be openly available for both peer-review and collaborative development. Many of the leading codes used in the low-temperature plasma community, including \textit{nonPDPSIM}, are not open to the public and thus cannot be held to the highest standard of review. In addition, without knowledge of the code's inner workings, reproduction of modeling results is more difficult. These are not criticisms of the code authors but rather a motivation for new code that \textit{is} openly available for public inspection and review as well as modification and customization.
\end{itemize}

\section{Experimental Design and Applications of Plasma-Liquid Interactions}
\label{sec:intro_apps}

\subsection{Maximizing Interactions between Plasmas and Liquids}
\label{sec:intro_epxt}

The above introduction and background hopefully motivated the need for additional modelling and simulation research on plasma-liquid systems. However, the experimental design of atmospheric pressure plasmas and their coupling with liquids is also critical for their applications in biomedicine, wastewater disinfection, agriculture, etc. A good review of atmospheric pressure plasma sources is given in \cite{tendero2006atmospheric}. In the article, the authors overview several popular discharges, including pulsed corona, DBD, pencil torches, ICP torches, the atmospheric pressure plasma jet (developed by Jeong et. al. \cite{park2001discharge}), and microwave discharges like the TIA developed by Moisan et. al. \cite{moisan1994atmospheric} In addition to the review by Tendero, Locke et. al. outline some discharges commonly used in plasma-liquid systems, focusing mostly on HV systems including coronas and gliding arc discharges. \cite{locke2006electrohydraulic} It is worthwile to note that many of the discharges used in plasma-liquid systems are filamentary in nature with water interaction spot sizes often around 1 mm in radius. For applications which require generation of large amounts of reactive species in the liquid, e.g. waste-water disinfection for example, such small areas of interaction are not ideal. The atmospheric pressure source developed by our group, described in detail in \cite{byrns2012vhf}, generates discharges roughly 2 cm in diameter with a plasma column that can span tens of centimeters. Such a large volume discharge that presents large surface areas for interaction is ideal for water treatment. This dissertation presents various geometric configurations for coupling the group's very high frequency (VHF) source to water. Along with exploiting the large volume of the discharge, we consider a configuration in which the water solution sits on top of the powered electrode, exposing the solution to increased fluxes of charged particles. This is somewhat similar to the gliding arc discharge shown in Figure 1K of \cite{locke2006electrohydraulic}, however, our glow discharge is steady and continuous as opposed to transient and filamentary. Consequently, we expect our powered electrode configuration, described in detail in \cref{sec:water_electrodes} to be state-of-the-art in terms of integrated charged, neutral, and UV fluxes from plasma to liquid.

\subsection{Fertigation}

Before the 1900s, nitrogen fixation occurred only naturally through lightning induced dissociation and reaction between atmospheric nitrogen and oxygen.  In 1903 the Norwegian team of Birkeland and Eyde attempted a copy of nature's fixation process when they flowed air through a thermal arc, creating nitric oxides which were then converted into nitric acid and finally a solid nitrate salt. \cite{kogelschatz2004atmospheric} Because of the intense energy requirements-17 kWh/kg nitric acid-the Birkeland-Eyde process was gradually replaced in Norway by the Haber and Ostwald processes.\cite{bakken1994high} In more recent years, generation of nitrates and nitrites in aqueous solution has been demonstrated with multiple non-thermal air discharges, including gliding arc, corona, and DBD.\cite{lelievre1995electrolysis,locke2006electrohydraulic,traylor2011long}

In agreement with the solution chemistries produced by the preceding non-thermal discharges, our VHF large volume glow discharge has demonstrated the ability to infuse nitrates into an aqueous medium, motivating a study of the impact of plasma activated water (PAW) on some traditional plant crops, outlined in detail in \cref{sec:fertigation}. This research contributes to the early work of Birkeland and Eyde and more contemporary work into agricultural applications of plasmas. There is significant literature on exposure of plant seeds either directly to the discharge or the discharge afterglow. Sera et. al.\cite{vsera2012various} investigated the effects of four plasma source types, including gliding arc, downstream microwave, and surface DBD (SDBD) on the growth of buckwheat seeds. They found that gliding arc improved seed growth while SDBD in close proximity to the seeds inhibited growth. Bormashenko et. al.\cite{bormashenko2012cold} found that an RF plasma generated under vacuum conditions increased the wettability of seed surfaces and seed germination rates. Zhou et. al.\cite{zhou2011introduction} used the afterglow of a DBD to increase seed growth in tomatoes while \cite{huang2010effect} and \cite{filatova2012fungicidal} have examined plasmas for improving wheat seed germination and eliminating fungus in grains and legumes respectively. While the literature for plasma treatment of seeds is extensive, the literature for using plasmas as a long-term aid in plant growth is less so, perhaps because of the historical failure of Birkeland and Eyde. A couple of recent studies have been published, however. Takaki et. al.\cite{takaki2013improvements} developed a unique system in a which a high voltage pulsed electrohydraulic discharge was used to both eliminate bacteria in recycled fertilizer water and to generate nitric acid, itself a fertilizer. They were able to demonstrate a significant increase in plant growth. Park et. al.\cite{park2013reactive} examined thermal spark discharge, gliding arc, and transferred arcs for generation of nitrogen species in water and subsequent application to a variety of plants. They found that the non-thermal gliding arc discharge produced the most promising plant growth results. However, in a study with radishes, banana peppers, and tomatoes, gliding arc treated water and a spring water control produced growth rates that were within experimental error of each other. To build on the work of these pioneering studies, we present a unique low-voltage large-volume glow discharge capable of generating PAW which has a statistically significant positive effect on the growth of radishes, marigolds, and tomatoes.

\subsection{Pollutant Remediation in Wastewater}

\subsubsection{Dioxane}

1,4-dioxane, more commonly known as simply dioxane, has the chemical structure shown in \cref{fig:diox_struct}. \cite{dioxStruct} It is largely immune to conventional water treatment techniques. It is very hydrophilic and has a very low vapor pressure, so carbon adsorption and air stripping are not feasible. \cite{suh2004study} Additionally, dioxane is resistant to biotransformations. \cite{suh2004study} With conventional techniques unable to degrade the chemical, a class of treatments known as Advanced Oxidative Techniques (AOTs) must be used. A typical AOT process may employ ozone, hydrogen peroxide, or other hydroxyl producing precursors. The general mechanism for hydroxyl production using H$_2$O$_2$ as a precursor is homolytic bond cleavage of the O-O bond with UV light: \cite{audenaert2011application}

\begin{equation}
  H_2O_2 + UV \rightarrow 2OH
  \label{eq:H2O2}
\end{equation}

The mechanism for OH production via O$_3$ is a little more complex: \cite{beltran2003ozone}

\begin{gather}\label{eq:O3}
  O_3 + OH^- \rightarrow HO_2^- + O_2\\
  O_3 + HO_2^- \rightarrow HO_2 + O_3^-\\
  O_3^- + H^+ \rightarrow HO_3\\
  HO_3 \rightarrow OH + O_2
\end{gather}

Photocatalytic oxidation with titanium dioxide (TiO$_2$) is another AOT that has received considerable attention in the literature. \cite{wantala2011visible,wantala2011visible,pelaez2012review,lee2013hybridization} OH production from TiO$_2$ and UV is theorized to proceed through the following steps: \cite{beltran2003ozone}

\begin{gather}\label{eq:TiO2}
  TiO_2 + UV \rightarrow e^- + h^+\\
  Ti(IV) + H_2O \rightleftharpoons Ti(IV)-H_2O\\
  Ti(IV)-H_2O + h^+ \rightleftharpoons Ti(IV)-OH + H^+
\end{gather}

where e$^-$ here represents an excited electron and h$^+$ an electron gap. Both ozone and hydrogen peroxide are readily created by atmospheric pressure discharges in contact with liquids depending on the power density of the discharge. \cite{Pavlovich2013g} Additionally UV and OH, created at the interface from impinging gas processes and in the bulk from decomposition of longer lived species, are expected to be formed in abundance. \cite{Tian2014} Thus plasmas are likely to be a suitable candidate for treating aqueous dixoane solutions. This is explored using the VHF source in \cref{sec:dioxane}.

\begin{figure}[htbp]
  \centering
  \includegraphics{1-4-Dioxane.eps}
  \caption{Chemical structure of dioxane.}
  \label{fig:diox_struct}
\end{figure}

Once OH is generated in solution, degradation of contaminants is expected to proceed through either hydrogen abstraction or electrophilic addition. This is illustrated through attack on benzene in \cref{fig:benzene_attack}. \cite{solarchem} It is speculated that as long as there are sufficient OH radicals, contaminant degradation will proceed until all fragments are converted into small, stable, terminal species like H$_2$O and CO$_2$.

\begin{figure}[htpb]
  \centering
  \includegraphics[width=.9\textwidth]{Mechanism_of_OH_with_benzene.png}
  \caption{OH radical attack on benzene}
  \label{fig:benzene_attack}
\end{figure}

\subsubsection{PFOS}

Perfluorooctane sulfonate (PFOS), like dixoane, is another persistent environmental pollutant (structure shown in \cref{fig:PFOS_struct}). \cite{pfosStruct} PFOS is an essential chemical for photolithographic processes and thus is extensively used in the semiconductor industry; it has no known substitutes. \cite{tang2006use} It is similarly resistant to conventional wastewater treatment techniques and is in addition resistant to AOTs because of its slow rate of reaction with hydroyxl radicals. \cite{cheng2008sonochemical} Perfluorochemicals like PFOS can theoretically be treated using techniques like activated carbon, nanofiltration, and reverse osmosis; however, the effectiveness of these treatments on wastewater can be significantly impaired by other contaminants in the water matrix. \cite{tang2006use,tang2007effect} Tang et. al. observed that adding isopropyl alcohol, another chemical used ubiquitously in the semiconductor industry, to solution had a detrimental effect on reverse osmosis treatment of PFOS. \cite{tang2006use}

A 2008 study showed that PFOS levels of 90 parts per billion (equivalent to 90 $\mu$g/L for aqueous solutions) compromised immune systems in male mice. \cite{betts2008chemical} The study concludes by saying that human immune systems could be compromised by similar PFOS levels. That $\mu$g/L levels of PFOS may cause negative health effects presents unique problems for degradation researchers. It is common for studies to use tens of mg/L initial contaminant concentrations when testing degradation techniques. In general treatment efficacy will be lower at lower contaminant concentrations. Treatment techniques that show promise include photochemical decomposition through use of persulfate ion \cite{hori2005efficient}, reduction using zerovalent iron \cite{hori2006efficient}, and acoustic cavitation. \cite{moriwaki2005sonochemical,cheng2008sonochemical} The relative intertness of PFOS to OH presents a challenge to plasma treatment. However, as demonstrated in \cref{sec:PFOS}, the water electrode configuration with the VHF source is able to efficiently remove PFOS at initial concentrations of 50 $\mu$g/L. See \cref{sec:PFOS} for details.

\begin{figure}[htbp]
  \centering
  \includegraphics{Perfluorooctanesulfonic_acid_structure.eps}
  \caption{Chemical structure of PFOS}
  \label{fig:PFOS_struct}
\end{figure}


\section{Dissertation Outline}

This disseration is laid out in the following way. \Cref{chap:basic_science} outlines modelling work investigating fundamental basic science questions in plasma-liquid systems. \Cref{sec:plasfree_model} addresses the questions of \cref{itm:advection} above; \cref{sec:plasliq} addresses \cref{itm:interface}. \Cref{chap:zapdos} describes the functionality of our code Zapdos, the need for which was outlined in \cref{sec:intro_plasliq}. Additionally, a description of the code added to the MOOSE framework itself allowing coupling of theplasma and liquid domains is given in \cref{sec:moose}. A description of the various experimental configurations explored for optimizing plasma-liquid interactions is given in \cref{chap:expt_opt}. Finally, our research on applications of plasma-liquid systems, including fertigation and remediation of wastewater contaminants like 1,4-dioxane and PFOS, is presented in \cref{chap:applications}.
